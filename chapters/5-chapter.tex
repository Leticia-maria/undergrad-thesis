\chapter{Conclusão}
% ----------------------------------------------------------

Por meio deste trabalho, foi apresentado o \textit{Balmy.jl}, uma aplicação \textit{web} capaz de auxiliar no cálculo de orbitais moleculares usando o método de Hueckel estendido. Comprovou-se, através dos resultados obtidos, que as energias dos orbitais moleculares foram coerentes com a literatura original de Roald Hoffmann, criador da metodologia \gls{EHMO}. Os tempos de execução dos cálculos em comparação a um código já existente (\gls{YAeHMOP}) foram extremamente satisfatórios, com ganhos de 12.5\% em termos de desempenho e performance conforme aumentamos o número de bases no sistema (no caso, foi testado o poletileno). Isso ocorre devido às otimizações que a linguagem Julia possui para implementação de equações matemáticas em relação a outras linguagens, como C, por exemplo. Além disso, outros ganhos foram observados no uso do \textit{Balmy.jl} se comparado ao \gls{YAeHMOP}, como a implementação com interface gráfica, o acesso via navegador \textit{web}, e os métodos de avaliação de compostos aromáticos (que não estão presentes no \gls{YAeHMOP}). Tudo isso torna a experiência do usuário ainda mais completa quando ele utiliza o \textit{Balmy.jl}.

Além dos cáculos via \gls{EHMO} feitos no \textit{software}, os índices de quantificação da aromaticidade sob o ponto de vista geométrico também mostraram resultados que estão em bom acordo com a literatura, comprovando que toda a implementação foi realizada adequadamente. A abordagem estatística do índice de aromaticidade mostra como o \gls{HOMA} pode ser adotado enquanto o melhor descritor de aromaticidade geométrica, pois ele é baseado em uma fundamentação físico-química das energias e forças de ligação obtidas pela teoria do oscilador harmônico. Apesar disso, o índice \gls{HOMA} não é um bom descritor de aromaticidade para sistemas heteroaromáticos. Nesse sentido, o \textit{Balmy.jl} também contém o \gls{rHOMA} e o \gls{HOMED} implementados para análise de sistemas heterocíclicos aromáticos. De acordo com os resultados mostrados, o \gls{HOMED} é um índice de heteroaromaticidade mais adequado do que o \gls{rHOMA}, devido às parametrizações do comprimento ótimo de ligação levando em conta sistemas aromáticos mínimos, como a metilamina e metilimina, por exemplo.

E foi nesse contexto do estudo da aromaticidade em diferentes famílias de compostos que surgiu o \textit{Balmy.jl}, uma ferramenta publicada na \textit{web} com o objetivo de auxiliar na interpretação eletrônica e geométrica desse fenômeno. O \textit{software} está disponível de forma gratuita, e por ser acessível tal qual uma página na \textit{internet}, não apresenta os mesmos problemas de compatibilidade do que uma aplicação \textit{desktop} com relação a diferentes sistemas operacionais. Além disso, o usuário não precisará se incomodar com atualizações em sua máquina local, pois tudo está hospedado em um servidor remoto, onde as requisições e cálculos são processados. Além disso, a implementação com a interface gráfica permite aos usuários menos experientes utilizar o \textit{Balmy.jl} de maneira extremamente fácil, sem nenhum requisito quanto a conhecimento de linguagens de programação ou códigos de linha de comando. Isso é novo com a relação a outros \textit{softwares} existentes dentro do mundo da química computacional, que não acumulam em si todas as vantagens que o \textit{Balmy.jl} demonstra considerando os métodos implementados.