\chapter{Conclusão}
% ----------------------------------------------------------

Por meio deste trabalho, foi apresentado o \textit{Balmy.jl}, uma aplicação \textit{web} capaz de auxiliar no cálculo de orbitais moleculares por meio do método de Hueckel estendido. Comprovou-se, através dos resultados, que as energias obtidas para os orbitais moleculares foram coerentes com a literatura original de Roald Hoffmann, criador da metodologia \gls{EHMO}. Os tempos de execução dos cálculos em comparação a um código já existente (\gls{YAeHMOP}) foram extremamente satisfatórios, com ganhos de 12.5\% em termos de desempenho e performance conforme aumentamos o número de bases no sistema. Isso ocorre devido às otimizações que a linguagem Julia possui para implementação de equações matemáticas em relação a outras linguagens, como C, por exemplo.

Os índices de quantificação da aromaticidade sob o ponto de vista geométrico também mostraram resultados que estão em bom acordo com a literatura. A abordagem estatística do índice de aromaticidade mostra como o \gls{HOMA} pode ser adotado enquanto o melhor descritor de aromaticidade geométrica, pois ele é baseado em uma fundamentação físico-química das energias e forças de ligação obtidas pela teoria do oscilador harmônico. Apesar disso, o índice \gls{HOMA} não é um bom descritor de aromaticidade para sistemas heteroaromáticos. Nesse sentido, o \textit{Balmy.jl} também contém o \gls{rHOMA} e o \gls{HOMED} implementados para análise de sistemas heterocíclicos aromáticos. De acordo com os resultados mostrados, o \gls{HOMED} é um melhor índice de heteroaromaticidade do que o \gls{rHOMA}, devido às parametrizações levando em conta sistemas aromáticos mínimos.