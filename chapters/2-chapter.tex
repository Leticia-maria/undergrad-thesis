% ----------------------------------------------------------
\chapter{Metodologia
}\label{cap:desenvolvimento}
% ----------------------------------------------------------
\section{Interface gráfica}

A interface gráfica publicada no navegador foi elaborada utilizando CSS, JavaScript e HTML. 

\section{Representação tridimensional}

As representações tridimensionais moleculares foram feitas com Three.js, uma biblioteca que usa WebGL para desenhar (na realidade, renderizar) objetos 3D de maneira facilitada. No caso das moléculas, utilizamos modelos conhecidos, como o de bola (átomos) e bastões (ligações químicas) (do inglês \textit{ball and stick}), usado para exibir geometrias de moléculas em 3D. 

As esferas são comumente coloridas seguindo a convenção CPK (Corey-Pauling-Koltun), que se popularizou por distinguir bem os elementos químicos nos modelos moleculares. Em 1952, Corey e Pauling publicaram uma descrição dos modelos de preenchimento de espaço de proteínas e outras biomoléculas que eles haviam construído na Caltech \autocite{Corey1953}. Tal proposição foi melhorada em 1965, por Koltun, que estendeu essa técnica aos halogênios e metais \autocite{Crossland2004-ll}.

Várias das cores do CPK referem-se mnemonicamente às cores dos elementos puros ou compostos notáveis. Por exemplo, o hidrogênio é um gás incolor, o carbono como carvão vegetal, grafite ou coque é preto, o enxofre em pó é amarelo, o cloro é um gás esverdeado, o bromo é um líquido vermelho escuro, o iodo no éter é violeta, o fósforo amorfo é vermelho, a ferrugem é vermelho alaranjado escuro, etc. Para algumas cores, tais como as de oxigênio e nitrogênio, a inspiração é menos clara. Talvez o vermelho para oxigênio seja inspirado pelo fato de que o oxigênio é normalmente necessário para a combustão ou que o químico que contém oxigênio no sangue, hemoglobina, é vermelho vivo, e o azul para nitrogênio pelo fato de que o nitrogênio é o principal componente da atmosfera terrestre, que aos olhos humanos parece ser de cor azul céu.

É provável que as cores do CPK tenham sido inspiradas por modelos no século XIX. Em 1865, August Wilhelm von Hofmann, em uma palestra no \textit{Royal Institution} em Londres, estava usando modelos feitos de bolas de \textit{croquet} para ilustrar a valência, então ele usou as bolas coloridas disponíveis para ele. (Na época, o \textit{croquet} era o esporte mais popular na Inglaterra, portanto, as bolas eram abundantes). \textit{On the Combining Power of Atoms}, \textit{Chemical News}, 12 (1865, 176-9, 189, afirma que:

\begin{citacao}
Hofmann, em uma palestra dada na Royal Institution em abril de 1865 fez uso de bolas de croquete de diferentes cores para representar vários tipos de átomos (por exemplo, preto de carbono, branco de hidrogênio, verde de cloro, vermelho de oxigênio 'ardente', azul nitrogênio)\autocite{Crossland2004-ll}.
\end{citacao}

\section{Método de Hueckel Estendido}

O \gls{HMO} foi inicialmente proposto por Erich Hueckel em 1930 \autocite{Hckel1931} como uma alternativa para calcular orbitais moleculares como combinações lineares de orbitais atômicos\autocite{Coulson1978-ot}. Tal teoria prediz os orbitais moleculares de elétrons $\pi$ em moléculas com estrutura eletrônica deslocalizada., tal como o etileno, benzeno, butadieno e piridina. Isso nos fornece uma base teórica para fundamentar a regra de Hueckel, responsável por enunciar que moléculas cíclicas, planares ou espécies químicas iônicas com $4n + 2$ elétrons $\pi$ são aromáticos.

Esse modelo baseia-se no fato de que a interação entre dois orbitais atômicos (sobreposição) produz um orbital molecular ligante mais estável (energia mais baixa), e um orbital molecular antiligante, menos estável do que aqueles que o geraram. Dessa forma, o número de orbitais moleculares novos é igual ao número de orbitais atômicos envolvidos (a combinação linear). A estabilidade relativa ou as energias dos orbitais moleculares em um polieno completamente conjugado, cíclico, planar pode ser predita pela teoria de Hueckel. Tomando o benzeno como exemplo, é possível ilustrar, através de um círculo de Frost (\autoref{fig:M1}), os níveis de energia dos seus orbitais moleculares de fronteira.

\begin{figure}[htb]
	\caption{\label{fig:M1} Círculo de Frost para o benzeno.}
	\begin{center}
		\includegraphics[width=0.70\textwidth]{images/figM.png}
	\end{center}
	\fonte{Autor(a).}
\end{figure}

%%% TODO : FROST DIAGRAM

Posteriormente, essa teoria foi estendida para tratar moléculas com heteroátomos (aqueles diferentes do carbono e hidrogênio)\autocite{Liwschitz1963}. Uma mudança ainda maior foi feita por Roald Hoffmann \autocite{Hoffmann1963}, que desenvolveu o método \gls{EHMO}, o qual inclui todos os elétrons da valência (inclusive elétrons $\sigma$) no cálculo das energias dos orbitais moleculares. Esse método é classificado como semiempírico, ou seja, utiliza dados experimentais para facilitar o processo de cálculo das integrais que acessam a interação elétron-elétron para contabilizar os efeitos de correlação eletrônica na estrutura química.

Como o objetivo do trabalho é desenvolver uma ferramenta rápida e eficiente para ser utilizada dentro no navegador de internet, tal método foi escolhido pelo seu baixo custo computacional de operação, devido às aproximações das matrizes Hamiltonianas pelo teorema de Koopman, responsável por aproximar os valores de $H_{ii}$ pelas energias de ionização. 

\section{Otimização de geometria}

Predizer o arranjo mais estável dos átomos em uma molécula é uma das mais importantes tarefas na química quântica computacional. Essencialmente, este é um problema de otimização onde a energia total da molécula é minimizada com relação às posições dos núcleos atômicos. A geometria molecular obtida desse cálculo é, dessa forma, um ponto de partida para inúmeras simulações de propriedades moleculares. Se a geometria não é acurada, então quaisquer cálculos que derivam dele também podem ser espúrios.

Uma vez que os núcleos são muito mais pesados do que os elétrons, nós podemos tratá-los como partículas pontuais associadas às suas respectivas posições. A partir disso, é possível afirmar que a energia da molécula $E(x)$.

\section{Teoria de grafos}

\begin{figure}[htb]
	\caption{\label{fig:M2} Exemplo ilustrativo de um grafo cíclico não direcionado. Os pontos em azul representam os seis vértices que se conectam através das linhas pretas correspondentes às arestas. O equivalente à esquerda é um anel benzênico de Kekulé, com seis átomos de carbono ocupando os nodos de um ciclo hexagonal.}
	\begin{center}
		\includegraphics[width=0.65\textwidth]{images/grafo(2).png}
	\end{center}
	\fonte{Autor(a).}
\end{figure}

\begin{figure}[htb]
\caption{\label{fig:graphEnumerated}Representação do grafo mostrado na \autoref{fig:graph} com nodos enumerados sequencialmente de 1-6.}
	\begin{center}
		\includegraphics[width=0.5\textwidth]{images/graphEnumerated.png}
	\end{center}
	\fonte{Autor(a)}
\end{figure}

\begin{figure}[htb]
\caption{\label{fig:DFS} Representação esquemática do algoritmo DFS (Algoritmo \ref{alg:1}). Todos os nodos adjacentes são visitados até que sejam marcados como visitados (cinza). O ciclo é encontrado quando o último nodo é igual ao nodo raiz.}
	\begin{center}
		\includegraphics[width=0.75\textwidth]{images/DFS.png}
	\end{center}
	\fonte{Autor(a)}
\end{figure}

\section{Critérios quantitativos da aromaticidade}