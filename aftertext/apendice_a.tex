% ----------------------------------------------------------
\chapter{Método de Hueckel}
% ----------------------------------------------------------

Considerando métodos semi-empíricos fundamentados na teoria \gls{HF}, pode-se citar o \gls{HMO}, no qual se aplicam as devidas aproximações resultantes dos conceitos e dados experimentais, pois estes tendem a ser intuitivos. Em alcenos e alcinos, os elétrons-$\pi$ estão presentes nos orbitais p, os quais são considerados como independentes da estrutura sigma dos orbitais híbridos e elétrons sigma. Funções de onda $\psi$ é dada pela \autoref{ap:eq:1}.

\begin{figure}[htb]
    \vspace{2\baselineskip}
\begin{equation}
    \label{ap:eq:1}
    \psi = \tikzmarknode{coefficient}{\highlight{red}{$a_1$}} \tikzmarknode{wavefunction}{\highlight{blue}{$\phi_1$}} + a_2\phi_2 + \dots + a_i \phi_i
\end{equation}
\begin{tikzpicture}[overlay,remember picture,>=stealth,nodes={align=left,inner ysep=1pt},<-]
    \path (wavefunction.north) ++ (-1.2em,1.5em) node[anchor=south east,color=blue!67] (scalep){\textit{função de onda}};
    \draw [color=blue!87](wavefunction.north) |- ([xshift=-3em,color=blue]scalep.south west);
    
    \path (coefficient.south) ++ (1.2em,-1.5em) node[anchor=north west,color=red!67] (scalep){\textit{coeficiente de contribuição}};
    \draw [color=red!87](coefficient.south) |- ([xshift=3em,color=red]scalep.north east);
\end{tikzpicture}
\end{figure}

Como somente os elétrons dos orbitais p estão contribuindo para a função de onda, a \autoref{ap:eq:1} pode ser escrita como a \autoref{ap:eq:2}.

\begin{figure}[htb]
    \vspace{2\baselineskip}
\begin{equation}
    \label{ap:eq:2}
    \psi = a_1 p_1 + a_2 p_2 + \dots + a_i p_i
\end{equation}
\end{figure}

Tomando o eteno como exemplo, pode-se afirmar que cada carbono contribui com um elétron para a ligação $\pi$, sendo $p_1$ e $p_2$ os elétrons dos átomos de carbono 1 e 2, cujas respectivas contribuições são ponderadas por $a_1$ e $a_2$. No caso dos elétrons p não hibridizados, os orbitais moleculares são formados pela \gls{LCAO} $p_1$ e $p_2$. Sobreposição entre orbitais atômicos podem ocorrer de forma simétrica (resultando em um orbital ligante) ou antissimétrica (formando um orbital antiligante).

\begin{figure}[htb]
    \vspace{2\baselineskip}
\begin{equation}
    \label{ap:eq:3}
    \psi =
    \begin{cases}
\tikzmarknode{simetric}{\highlight{blue}{$\psi^+$}} = a_1 p_1 + a_2 p_2 \\
\tikzmarknode{antissimetric}{\highlight{red}{$\psi^-$}} = a_1 p_1 - a_2 p_2
    \end{cases}
\end{equation}
\begin{tikzpicture}[overlay,remember picture,>=stealth,nodes={align=left,inner ysep=1pt},<-]
    \path (simetric.north) ++ (-1.2em,1.5em) node[anchor=south east,color=blue!67] (scalep){\textit{função simétrica}};
    \draw [color=blue!87](simetric.north) |- ([xshift=-3em,color=blue]scalep.south west);
    
    \path (antissimetric.south) ++ (1.2em,-1.5em) node[anchor=north west,color=red!67] (scalep){\textit{função antissimétrica}};
    \draw [color=red!87](antissimetric.south) |- ([xshift=3em,color=red]scalep.north east);
\end{tikzpicture}
\end{figure}

Usando a \autoref{ap:eq:4} (equação de Schroedinger) como base fundamental para avançar na discussão do cálculo dos valores de energia associados aos orbitais, pode-se aplicar o princípio variacional como uma aproximação necessária e conveniente para encontrar as funções de minimização da energia relativa aos orbitais no estado fundamental. Este método consiste em escolher uma função de onda inicial que dependa de um ou mais parâmetros, e encontrar os valores destes parâmetros para cada valor esperado onde a energia seja a menor possível. A função de onda obtida na substituição dos parâmetros pelos valores encontrados será uma aproximação do estado fundamental da função de onda, e o valor esperado de energia neste estado será majorante para a energia deste estado fundamental.

%%% TODO: explicar o princípio variacional 

\begin{figure}[htb]
    \vspace{2\baselineskip}
\begin{equation}
\label{ap:eq:4}
    \tikzmarknode{hamiltonian}{\highlight{red}{$\hat{H}$}} \psi = \tikzmarknode{energy}{\highlight{blue}{$E$}} \psi
\end{equation}
\begin{tikzpicture}[overlay,remember picture,>=stealth,nodes={align=left,inner ysep=1pt},<-]
    \path (energy.north) ++ (-1.2em,1.5em) node[anchor=south east,color=blue!67] (scalep){\textit{energia}};
    \draw [color=blue!87](energy.north) |- ([xshift=-3em,color=blue]scalep.south west);
    
    \path (hamiltonian.south) ++ (1.2em,-1.5em) node[anchor=north west,color=red!67] (scalep){\textit{operador}};
    \draw [color=red!87](hamiltonian.south) |- ([xshift=3em,color=blue]scalep.north east);
\end{tikzpicture}
\end{figure}

\begin{equation}
\label{ap:eq:5}
    \psi \hat{H} \psi = \psi^2 E
\end{equation}


Multiplicando ambos os lados por $\psi$, obtém-se a \autoref{ap:eq:5}. Aplicado a integral nessa expressão (mantendo a igualdade) e, em seguida, isolando a energia $E$, obtém-se a \autoref{ap:eq:6}.

\begin{equation}
\label{ap:eq:6}
    E = \frac{\displaystyle \int \psi \hat{H} \psi d\tau}{\displaystyle \int \psi^2 d\tau}
\end{equation}

Na \autoref{ap:eq:5}, a energia esperada vai ser superestimada em relação ao valor real. Desse modo, o resultado calculado vai ser minimizado através de um procedimento matemático submetido a um conjunto de funções de base. No caso do eteno (\ce{C_2 H_2}), podemos substituir o termo $\psi$ da \autoref{ap:eq:6} pela \autoref{ap:eq:2}, obtendo a \autoref{ap:eq:7}.

\begin{figure}[htb]
    \vspace{2\baselineskip}
\begin{equation}
\label{ap:eq:7}
    E = \frac{\displaystyle \int \tikzmarknode{LCAO}{\highlight{blue}{$(a_1 p_1 + a_2 p_2)$}} \hat{H} \highlight{blue}{$(a_1 p_1 + a_2 p_2)$} d\tau}{\displaystyle \int \highlight{blue}{$(a_1 p_1 + a_2 p_2)$}^2 d\tau}
\end{equation}
\begin{tikzpicture}[overlay,remember picture,>=stealth,nodes={align=left,inner ysep=1pt},<-]
    \path (LCAO.north) ++ (1.2em,1.5em) node[anchor=south west,color=blue!67] (scalep){LCAO \textit{do eteno}};
    \draw [color=blue!87](LCAO.north) |- ([xshift=3em,color=blue]scalep.south east);
\end{tikzpicture}
\end{figure}

Abrindo a expressão da \autoref{ap:eq:7}, é possível isolar as seguintes integrais:

\begin{figure}[htb]
    \vspace{2\baselineskip}
\begin{equation}
\label{ap:eq:8}
    \displaystyle \int (p_1 \hat{H} p_1) d\tau = \tikzmarknode{alpha}{\highlight{blue}{$\alpha$}} \\
\end{equation}
\begin{tikzpicture}[overlay,remember picture,>=stealth,nodes={align=left,inner ysep=1pt},<-]
    \path (alpha.north) ++ (-1.2em,1.5em) node[anchor=south east,color=blue!67] (scalep){\textit{integral de Coulomb}};
    \draw [color=blue!87](alpha.north) |- ([xshift=-3em,color=blue]scalep.south west);
\end{tikzpicture}
\end{figure}

A integral de Coulomb, mostrada na \autoref{ap:eq:8}, é o Hamiltoniano para a energia de Coulomb relativa a um elétron com uma função de onda $p_i$ no campo do átomo $i$ e influenciado pelo seu núcleo. Esse elétron não sofre efeito de nenhum outro núcleo. Essa aproximação funciona melhor quando os átomos das vizinhanças não possuem cargas elétricas. Ou seja, $\alpha$ é uma função da carga nuclear e do tipo de orbital. Por ser um termo atrativo, possui um valor negativo. 

\begin{figure}[htb]
    \vspace{2\baselineskip}
\begin{equation}
\label{ap:eq:9}
    \displaystyle \int (p_1 \hat{H} p_2) d\tau = \displaystyle \int (p_2 \hat{H} p_1) d\tau = \tikzmarknode{beta}{\highlight{blue}{$\beta$}}
\end{equation}
\begin{tikzpicture}[overlay,remember picture,>=stealth,nodes={align=left,inner ysep=1pt},<-]
    \path (beta.north) ++ (-1.2em,1.5em) node[anchor=south east,color=blue!67] (scalep){\textit{integral de troca (ou de \textbf{ressonância})}};
    \draw [color=blue!87](beta.north) |- ([xshift=-3em,color=blue]scalep.south west);
\end{tikzpicture}
\end{figure}

As integrais de sobreposição são dadas pelas expressões na \autoref{ap:eq:10}. Se $i=j$, então $S_{ii} = \displaystyle \int p_i p_i d \tau$ para os orbitais atômicos normalizados. Se $i \neq j $, a integral de sobreposição é igual a $0$ para os orbitais atômicos ortogonais. Ou seja, o valor de $S$ varia de $0$ a $1$ e representa uma medida da não-ortogonalidade dos orbitais. Uma vez que as funções $p$ dos orbitais são amplamente separadas no espaço e são independentes, espera-se que elas sejam ortogonais.

\begin{figure}[htb]
    \vspace{2\baselineskip}
\begin{equation}
\label{ap:eq:10}
\displaystyle \int (p_i p_j) d\tau =
\begin{cases}
 \tikzmarknode{s11}{\highlight{blue}{$S_{11}$}} = \displaystyle \int (p_1 p_1) d\tau = \highlight{blue}{$S_{22}$} = \displaystyle \int (p_2 p_2) d\tau \\
 \tikzmarknode{s12}{\highlight{blue}{$S_{12}$}} = \displaystyle \int (p_2 p_1) d\tau = \highlight{blue}{$S_{21}$} = \displaystyle \int (p_1 p_2) d\tau
\end{cases}
\end{equation}
\begin{tikzpicture}[overlay,remember picture,>=stealth,nodes={align=left,inner ysep=1pt},<-]
    \path (s11.north) ++ (1.2em,1.5em) node[anchor=south west,color=blue!67] (scalep){\textit{integral de sobreposição ($i = j$)}};
    \draw [color=blue!87](s11.north) |- ([xshift=3em,color=blue]scalep.south east);
    
    \path (s12.south) ++ (1.2em,-1.5em) node[anchor=north west,color=blue!67] (scalep){\textit{integral de sobreposição ($i \neq j$)}};
    \draw [color=blue!87](s12.south) |- ([xshift=3em,color=blue]scalep.north east);
\end{tikzpicture}
\end{figure}

Com as simplificações feitas anteriormente, é possível reescrever a expressão que calcula a energia para a molécula de eteno de acordo com a \autoref{ap:eq:12}.

\begin{figure}[htb]
    \vspace{2\baselineskip}
\begin{equation}
\label{ap:eq:12}
    E = \frac{\tikzmarknode{numerador}{\highlight{blue}{$a_1^2 \alpha + 2a_1 a_2 \beta + a_2^2 \alpha$}}}{\tikzmarknode{denominador}{\highlight{red}{$a_1^2 S_{11} + 2a_1 a_2 S_{12} + a_2^2 S_{22}$}}}
\end{equation}
\begin{tikzpicture}[overlay,remember picture,>=stealth,nodes={align=left,inner ysep=1pt},<-]
    \path (numerador.north) ++ (1.2em,1.5em) node[anchor=south west,color=blue!67] (scalep){\textit{numerador($N$)}};
    \draw [color=blue!87](numerador.north) |- ([xshift=3em,color=blue]scalep.south east);
    
    \path (denominador.south) ++ (1.2em,-1.5em) node[anchor=north west,color=red!67] (scalep){\textit{denominador($D$)}};
    \draw [color=red!87](denominador.south) |- ([xshift=3em,color=red]scalep.north east);
\end{tikzpicture}
\end{figure}

Desse modo, conhecendo $\alpha$, $\beta$ e $S$, a energia pode ser calculada. O critério de minimização em relação a alguns parâmetros é mostrado na \autoref{ap:eq:12}.

\begin{figure}[htb]
    \vspace{2\baselineskip}
\begin{equation}
\label{ap:eq:13}
    \frac{\partial E}{\partial a_1} = \frac{\partial E}{\partial a_2} = 0
\end{equation}
\end{figure}

Como alternativa, ao invés de variar a função teste para encontrar o valor mínimo de $E$ (\autoref{ap:eq:12}), pode-se variar os coeficientes lineares. Esse é um caso direto da busca pelo mínimo de uma função. Aplicando, então, a condição da \autoref{ap:eq:13}), obtém-se a \autoref{ap:eq:13a}.

\begin{figure}[htb]
    \vspace{2\baselineskip}
\begin{equation}
\label{ap:eq:13a}
    \frac{\partial E}{\partial a_1} = \frac{\tikzmarknode{nlinha}{\highlight{blue}{$N'$}}D - E \tikzmarknode{dlinha}{\highlight{red}{$D'$}}}{D^2}
\end{equation}
\begin{tikzpicture}[overlay,remember picture,>=stealth,nodes={align=left,inner ysep=1pt},<-]
    \path (nlinha.north) ++ (-1.2em,1.5em) node[anchor=south east,color=blue!67] (scalep){\textit{$2(a_1 \alpha + a_2 \beta)$}};
    \draw [color=blue!87](nlinha.north) |- ([xshift=-3em,color=blue]scalep.south west);
    
    \path (dlinha.north) ++ (1.2em,1.5em) node[anchor=south west,color=red!67] (scalep){\textit{$2(a_1 S_{11} + a_2 S_{12})$}};
    \draw [color=red!87](dlinha.north) |- ([xshift=3em,color=red]scalep.south east);
\end{tikzpicture}
\end{figure}


\chapter{Método de Hueckel Estendido}

O \gls{EHMO} surgiu como um aprimoramento do procedimento enunciado por Hueckel, considerando todos os elétrons da valência em um cálculo de orbitais moleculares (\autoref{ap:eq:14}). Ou seja, a partir disso, pode-se calcular as barreiras de energia para rotacionar as ligações, e mesmo determinar as energias e estruturas dos estados de transição para reações.

A função de onda eletrônica é considerada, nesse modelo, como um produto de funções de onda monoeletrônicas:

\begin{figure}[htb]
    \vspace{2\baselineskip}
\begin{equation}
    \label{ap:eq:14}
    \psi_{\textbf{total}} = \phi_{\textbf{núcleo}} + \tikzmarknode{hmo}{\highlight{red}{$\phi_{\textbf{valência}}$}}
\end{equation}
\begin{tikzpicture}[overlay,remember picture,>=stealth,nodes={align=left,inner ysep=1pt},<-]
    \path (hmo.north) ++ (1.2em,1.5em) node[anchor=south west,color=red!67] (scalep){\textit{Hueckel estendido}};
    \draw [color=red!87](hmo.north) |- ([xshift=3em,color=red]scalep.south east);
\end{tikzpicture}
\end{figure}

\begin{figure}[htb]
    \vspace{2\baselineskip}
\begin{equation}
    \phi_{\textbf{valência}} = \psi_1 (1) \psi_2 (2) \psi_3 (3) \dots \tikzmarknode{j}{\highlight{red}{$\psi_j$}} (\tikzmarknode{n}{\highlight{blue}{$n$}})
\end{equation}
\begin{tikzpicture}[overlay,remember picture,>=stealth,nodes={align=left,inner ysep=1pt},<-]
    \path (j.north) ++ (1.2em,1.5em) node[anchor=south west,color=red!67] (scalep){$\displaystyle \psi_j = \sum_{r=1}^N c_{jr} \phi_r$};
    \draw [color=red!87](j.north) |- ([xshift=3em,color=red]scalep.south east);
    
    \path (n.south) ++ (1.2em,-1.5em) node[anchor=north west,color=blue!67] (scalep){\textit{$n^o$ de elétrons}};
    \draw [color=blue!87](n.south) |- ([xshift=3em,color=blue]scalep.north east);
\end{tikzpicture}
\end{figure}

\noindent onde $n$ é o número de elétrons e $j$ identifica o orbital molecular. Novamente, cada um dos \gls{MOs} é dada pela LCAO. O conjunto de orbitais aqui definidos são chamados \textit{conjunto mínimo de base}, pois contém somente as informações dos orbitais atômicos para a camada de valência dos átomos. 

\begin{equation}
    HC = SCE
\end{equation}

\noindent onde $H$ é uma matriz quadrada contendo o $H_{rs}$, integrais de energia de um único elétron, e $C$ é a matriz dos coeficientes relativos aos orbitais atômicos. Cada coluna em $C$ é $C'$, que define um orbital molecular em termos das funções de base. Na teoria molecular de Hueckel estendido, a sobreposição não é negligenciado, e $S$ é a matriz das integrais de sobreposição. Essas matrizes são quadradas, possuem o tamanho igual ao número de orbitais atômicos usados na combinação linear. Para qualquer cálculo desse tipo, deve-se construir essas matrizes e então encontrar os autovalores (energias orbitais) e autovetores (coeficientes que definem o orbital molecular em termos das funções de base).

Os elementos da matriz $H$ são atribuídos a partir de dados experimentais, uma vez que o método em questão é semi-empírico. Os elementos de matriz Hamiltoniana fora de diagonal são dados por uma aproximação devido a Wolfsberg e Helmholz
que os relaciona com os elementos diagonais e o elemento da matriz de sobreposição.

\begin{equation}
    H_{ij} = \frac{1}{2} K(H_{ii} + H_{jj}) S_{ij}
\end{equation}

De acordo com essa expressão, a energia deveria ser proporcional à energia dos orbitais atômicos, e será tão maior quanto maior a sobreposição desses orbitais. A contribuição desses efeitos com a energia é escalada pelo fator $K$. Hoffmann \autocite{Hoffmann1963} atribuiu o valor de 1.75 para $K$, após um estudo do efeito desse parâmetro nas energias dos orbitais ocupados do etano. Os valores de $H_{ii}$ são escolhidos como um estado de potenciais de ionização com o valor negativo, indicando ligação. Isso ocorre em função do teorema de Koopman\autocite{Koopmans1934}, cujo enunciado determina que a primeira energia de ionização de um sistema molecular é o oposto da energia atribuída ao \gls{HOMO}, uma vez que. no contexto do \gls{RHF}, considera-se que os orbitais do íon são idênticos àqueles da molécula neutra (segundo a aproximação do orbital \textit{congelado}). 

