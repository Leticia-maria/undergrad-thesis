% ----------------------------------------------------------
\chapter{Método de Hueckel}
% ----------------------------------------------------------

Considerando métodos semiempíricos fundamentados na teoria \gls{HF}, utilizam-se as devidas aproximações resultantes dos conceitos e dados experimentais, pois estes tendem a ser intuitivos. Em alcenos e alcinos, os elétrons-$\pi$ estão presentes nos orbitais-p, os quais são considerados como independentes da estrutura sigma dos orbitais híbridos e elétrons sigma. Funções de onda $\Psi$ é dada pela \autoref{ap:eq:1}.

\begin{figure}[htb]
    \vspace{2\baselineskip}
\begin{equation}
    \label{ap:eq:1}
    \Psi =  a_1\tikzmarknode{wavefunction}{\highlight{blue}{$\phi_1$}} + \tikzmarknode{coefficient}{\highlight{red}{$a_2$}}\phi_2 + \dots + a_i \phi_i
\end{equation}
\begin{tikzpicture}[overlay,remember picture,>=stealth,nodes={align=left,inner ysep=1pt},<-]
    \path (wavefunction.north) ++ (-1.2em,1.5em) node[anchor=south east,color=blue!67] (scalep){\textbf{contribution coefficient}};
    \draw [color=blue!87](wavefunction.north) |- ([xshift=-3em,color=blue]scalep.south west);
    
    \path (coefficient.south) ++ (1.2em,-1.5em) node[anchor=north west,color=red!67] (scalep){\textbf{eletronic wavefunction}};
    \draw [color=red!87](coefficient.south) |- ([xshift=3em,color=red]scalep.north east);
\end{tikzpicture}
\end{figure}