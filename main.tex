% ------------------------------------------------------------------------
% ------------------------------------------------------------------------
% Modelo UFSC para Trabalhos Academicos (tese de doutorado, dissertação de
% mestrado) utilizando a classe abntex2
%
% Autor: Alisson Lopes Furlani
% 	Modificações:
%	- 27/08/2019: Alisson L. Furlani, add pacote 'glossaries' para listas
%   - 06/11/2019: Luiz-Rafael Santos, modifica para Trabalho de Conclusão de Curso
% ------------------------------------------------------------------------
% ------------------------------------------------------------------------

\documentclass[
	% -- opções da classe memoir --
	12pt,				% tamanho da fonte
	%openright,			% capítulos começam em pág ímpar (insere página vazia caso preciso)
	oneside,			% para impressão no anverso. Oposto a twoside
	a4paper,			% tamanho do papel. 
	% -- opções da classe abntex2 --
	chapter=TITLE,		% títulos de capítulos convertidos em letras maiúsculas
	section=TITLE,		% títulos de seções convertidos em letras maiúsculas
	%subsection=TITLE,	% títulos de subseções convertidos em letras maiúsculas
	%subsubsection=TITLE,% títulos de subsubseções convertidos em letras maiúsculas
	% -- opções do pacote babel --
	english,			% idioma adicional para hifenização
	%french,				% idioma adicional para hifenização
	%spanish,			% idioma adicional para hifenização
	brazil				% o último idioma é o principal do documento
	]{abntex2}

\usepackage{setup/ufscthesisA4-alf}
\usepackage{sansmathfonts}
\usepackage[T1]{fontenc}
\renewcommand*\familydefault{\sfdefault} %% Only if the base font of the document is to be sans serif
\usepackage[version=4]{mhchem}
\DeclareUnicodeCharacter{0301}{\'{e}}
\usepackage[dvipsnames]{xcolor}
\usepackage{tikz}
\usetikzlibrary{backgrounds}
\usetikzlibrary{arrows,shapes}
\usetikzlibrary{tikzmark}
\usetikzlibrary{calc}

\usepackage{amsmath}
\usepackage{booktabs}
\usepackage{amsthm}
\usepackage{amssymb}
\usepackage{mathtools, nccmath}
\usepackage{wrapfig}
\usepackage{comment}

% To generate dummy text
\usepackage{blindtext}
\usepackage[version=4]{mhchem}
% for tikz
\usepackage{tikz}
%\usetikzlibrary{trees}
\usetikzlibrary{arrows,shapes,positioning,shadows,trees,mindmap}
% \usepackage{forest}
\usepackage[edges]{forest}
\usetikzlibrary{arrows.meta}
\colorlet{linecol}{black!75}
\usepackage{xkcdcolors} % xkcd colors


% for colorful equation
\usepackage{tikz}
\usetikzlibrary{backgrounds}
\usetikzlibrary{arrows,shapes}
\usepackage{tcolorbox}
% for custom commands
\usepackage{xspace}

% table alignment
\usepackage{array}
\usepackage{ragged2e}
\usetikzlibrary{tikzmark}
\usetikzlibrary{calc}
% Commands for Highlighting text -- non tikz method
\newcommand{\highlight}[2]{\colorbox{#1!17}{$\displaystyle #2$}}
%\newcommand{\highlight}[2]{\colorbox{#1!17}{$#2$}}
\newcommand{\highlightdark}[2]{\colorbox{#1!47}{$\displaystyle #2$}}

% my custom colors for shading
\colorlet{mhpurple}{Plum!80}


% Commands for Highlighting text -- non tikz method
% ---

% Commands for Highlighting text -- non tikz method
\renewcommand{\highlight}[2]{\colorbox{#1!17}{#2}}
\renewcommand{\highlightdark}[2]{\colorbox{#1!47}{#2}}

% Some math definitions
\newcommand{\lap}{\mathrm{Lap}}
\newcommand{\pr}{\mathrm{Pr}}

\newcommand{\Tset}{\mathcal{T}}
\newcommand{\Dset}{\mathcal{D}}
\newcommand{\Rbound}{\widetilde{\mathcal{R}}}
% Filtering and Mapping Bibliographies
% ---
% Pacotes de citações
% ---
\usepackage{csquotes}
% \usepackage[backend = biber, style = abnt]{biblatex}
% FIXME Se desejar estilo numérico de citações,  comente a linha acima e descomente a linha a seguir.
% \usepackage[backend = bibtex, style = numeric-comp]{biblatex}
\usepackage[style=numeric-comp, autocite = superscript, doi=true, sorting=none, url=false, maxcitenames=3, maxbibnames=100, block=none, backref=true]{biblatex}

\setlength\bibitemsep{\baselineskip}
\DeclareFieldFormat{url}{Disponível~em:\addspace\url{#1}}
\NewBibliographyString{sineloco}
\NewBibliographyString{sinenomine}
\DefineBibliographyStrings{brazil}{%
	sineloco     = {\mkbibemph{S\adddot l\adddot}},
	sinenomine   = {\mkbibemph{s\adddot n\adddot}},
	andothers    = {\mkbibemph{et\addabbrvspace al\adddot}},
	in			 = {\mkbibemph{In:}}
}

%\usepackage{sansmathfonts}
%\usepackage[T1]{fontenc}
%\renewcommand*\familydefault{\sfdefault}
\addbibresource{aftertext/references.bib} % Seus arquivos de referências
\usepackage{xfrac,bigints}
% ---
\DeclareSourcemap{
	\maps[datatype=bibtex]{
		% remove fields that are always useless
		\map{
			\step[fieldset=abstract, null]
			\step[fieldset=pagetotal, null]
		}
		% remove URLs for types that are primarily printed
%		\map{
%			\pernottype{software}
%			\pernottype{online}
%			\pernottype{report}
%			\pernottype{techreport}
%			\pernottype{standard}
%			\pernottype{manual}
%			\pernottype{misc}
%			\step[fieldset=url, null]
%			\step[fieldset=urldate, null]
%		}
		\map{
			\pertype{inproceedings}
			% remove mostly redundant conference information
			\step[fieldset=venue, null]
			\step[fieldset=eventdate, null]
			\step[fieldset=eventtitle, null]
			% do not show ISBN for proceedings
			\step[fieldset=isbn, null]
			% Citavi bug
			\step[fieldset=volume, null]
		}
	}
}
% ---

% ---
% Informações de dados para CAPA e FOLHA DE ROSTO
% ---
% FIXME Substituir 'Nome completo do autor' pelo seu nome.
\autor{Letícia Maria Pequeno Madureira}
% FIXME Substituir 'Título do trabalho' pelo título da trabalho.
\titulo{\textit{Balmy.jl}: Desenvolvimento de \textit{Software} para Cálculos  de Aromaticidade}
% FIXME Substituir 'Subtítulo (se houver)' pelo subtítulo da trabalho.  
% Caso não tenha substítulo, comente a linha a seguir.
\subtitulo{Implementação com ambiente gráfico}
% FIXME Substituir 'XXXXXX' pelo nome do seu
% orientador.
\orientador{Prof. Dr. Giovanni Finoto Caramori}
% FIXME Se for orientado por uma mulher, comente a linha acima e descomente a linha a seguir.
% \orientador[Orientadora]{Nome da orientadora, Dra.}
% FIXME Substituir 'XXXXXX' pelo nome do seu
% coorientador. Caso não tenha coorientador, comente a linha a seguir.
% \coorientador{Prof. XXXXXX, Dr.}
% FIXME Se for coorientado por uma mulher, comente a linha acima e descomente a linha a seguir.
% \coorientador[Coorientadora]{XXXXXX, Dra.}
% FIXME Substituir 'XXXXXX' pelo nome do Coordenador do 
% programa/curso.
% \coordenador{Prof. XXXXXX, Dr.}
% FIXME Se for coordenadora mulher, comente a linha acima e descomente a linha a seguir.
\coordenador[Coordenadora]{Nome da Coordenadora, Dra.}
% FIXME Substituir '[ano da entrega]' pelo ano (ano) em que seu trabalho foi defendido.
\ano{2022}
% FIXME Substituir '[dia] de [mês] de [ano]' pela data em que ocorreu sua defesa.
\data{15 de julho de 2022}
% FIXME Substituir '[Cidade da defesa]' pela cidade em que ocorreu sua defesa.
\local{Florianópolis}
\instituicaosigla{UFSC}
\instituicao{Universidade Federal de Santa Catarina}
% FIXME Substituir 'Dissertação/Tese' pelo tipo de trabalho (Tese, Dissertação). 
\tipotrabalho{Trabalho de Conclusão de Curso}
% FIXME Substituir '[licenciado/bacharel] em [nome do título obtido]' pela grau adequado.
\formacao{Bacharel(a) em Química}
% FIXME Substituir '[licenciado/bacharel]' pelo nivel adequado.
\nivel{Bacharel(a)}
% FIXME Substituir 'Curso de Graduação em [XXXXXXXX]' pela curso adequado.
\programa{Curso de Graduação em Química}
% FIXME Substituir 'Campus XXXXXX ou Centro de XXXXXX' pelo campus ou centro adequado.
\centro{Campus Reitor João David Ferreira Lima ou Centro de Ciências Físicas e Matemáticas}
\preambulo
{%
\imprimirtipotrabalho~do~\imprimirprograma~do~\imprimircentro~da~\imprimirinstituicao~para~a~obtenção~do~título~de~\imprimirformacao.
}
% ---

% ---
% Configurações de aparência do PDF final
% ---
% alterando o aspecto da cor azul
\definecolor{blue}{RGB}{41,5,195}
% informações do PDF
\makeatletter
\hypersetup{
     	%pagebackref=true,
		pdftitle={\@title}, 
		pdfauthor={\@author},
    	pdfsubject={\imprimirpreambulo},
	    pdfcreator={LaTeX with abnTeX2},
		pdfkeywords={ufsc, latex, abntex2}, 
		colorlinks=true,       		% false: boxed links; true: colored links
    	linkcolor=black,%blue,          	% color of internal links
    	citecolor=black,%blue,        		% color of links to bibliography
    	filecolor=black,%magenta,      		% color of file links
		urlcolor=black,%blue,
		bookmarksdepth=4
}
\makeatother
% ---

% ---
% compila a lista de abreviaturas e siglas e a lista de símbolos
% ---

% Declaração das siglas
\siglalista{HF}{Hartree-Fock}
\siglalista{RHF}{Hartree-Fock Restrito}
\siglalista{MOs}{Orbitais Moleculares}
\siglalista{AOs}{Orbitais Atômicos}
\siglalista{LCAO}{Combinação Linear de Orbitais Atômicos}
\siglalista{HOMA}{Modelo de Aromaticidade baseado em Oscilador Harmônico}
\siglalista{HMO}{Método de Hueckel}
\siglalista{HOMO}{Highest Occupied Molecular Orbital}
\siglalista{EHMO}{Método de Hueckel Estendido}

% Declaração dos simbolos

% compila a lista de abreviaturas e siglas e a lista de símbolos
\makenoidxglossaries 

% ---

% ---
% compila o indice
% ---
\makeindex
% ---

% ----
% Início do documento
% ----
\begin{document}

% Seleciona o idioma do documento (conforme pacotes do babel)
%\selectlanguage{english}
\selectlanguage{brazil}

% Retira espaço extra obsoleto entre as frases.
\frenchspacing 

% Espaçamento 1.5 entre linhas
\OnehalfSpacing

% Corrige justificação
%\sloppy

% ----------------------------------------------------------
% ELEMENTOS PRÉ-TEXTUAIS
% ----------------------------------------------------------
% \pretextual %a macro \pretextual é acionado automaticamente no início de \begin{document}
% ---
% Capa, folha de rosto, ficha bibliografica, errata, folha de apróvação
% Dedicatória, agradecimentos, epígrafe, resumos, listas
% ---
% ---
% Capa
% ---
\imprimircapa
% ---

% ---
% Folha de rosto
% (o * indica que haverá a ficha bibliográfica)
% ---
\imprimirfolhaderosto*
% ---

% ---
% Inserir a ficha bibliografica
% ---
% http://ficha.bu.ufsc.br/
\begin{fichacatalografica}
	\includepdf{beforetext/Ficha_Catalografica.pdf}
\end{fichacatalografica}
% ---

% ---
% Inserir folha de aprovação
% ---
\begin{folhadeaprovacao}
	\OnehalfSpacing
	\centering
	\imprimirautor\\%
	\vspace*{10pt}		
	\textbf{\imprimirtitulo}%
	\ifnotempty{\imprimirsubtitulo}{:~\imprimirsubtitulo}\\%
	%		\vspace*{31.5pt}%3\baselineskip
	\vspace*{\baselineskip}
	%\begin{minipage}{\textwidth}
	% ~do~\imprimirprograma~do~\imprimircentro~da~\imprimirinstituicao~para~a~obtenção~do~título~de~\imprimirformacao.
	Este~\imprimirtipotrabalho~foi julgado adequado para obtenção do Título de ~\imprimirformacao~e aprovado em sua forma final pelo~\imprimirprograma. \\
		\vspace*{\baselineskip}
	\imprimirlocal, \imprimirdata. \\
	\vspace*{2\baselineskip}
	\assinatura{\OnehalfSpacing\imprimircoordenador \\ \imprimircoordenadorRotulo~do Curso}
	\vspace*{2\baselineskip}
	\textbf{Banca Examinadora:} \\
	\vspace*{\baselineskip}
	\assinatura{\OnehalfSpacing\imprimirorientador \\ \imprimirorientadorRotulo}
	%\end{minipage}%
	\vspace*{\baselineskip}
	\assinatura{Prof. Dr. Luis Herique da Silveira Lacerda\\
	Avaliador(a) \\
	Universidade Federal de Santa Catarina}

	\vspace*{\baselineskip}
	\assinatura{Prof. Dr. Thiago Ferreira da Conceição\\
	Avaliador(a) \\
	Universidade Federal de Santa Catarina}


\end{folhadeaprovacao}
% ---

% ---
% Dedicatória
% ---
\begin{dedicatoria}
	\vspace*{\fill}
	\noindent
	\begin{adjustwidth*}{}{5.5cm}     
		Este trabalho aos meus pais, meu orientador, e aos meus colegas de laboratório.
	\end{adjustwidth*}
\end{dedicatoria}
% ---

% ---
% Agradecimentos
% ---
\begin{agradecimentos}
	Gostaria de tecer um agradecimento primeiro à Universidade Federal de Santa Catarina e ao ensino público de qualidade (e que assim permaneça!), aos meus pais, Glauber e Maria, por sempre terem me dado todo o suporte necessário para o aprendizado independentemente de qualquer entrave, e à minha irmã, Larissa, por sempre ter me animado nos períodos difíceis, e aos meus 8 gatos, principalmente o Chokito, por ser o companheiro de estudos mais fiel que alguém poderia ter.

Não poderia deixar de citar todos os professores do Departamento de Química, que contribuíram tanto para a minha formação acadêmica, particularmente o meu orientador Giovanni F. Caramori, que me ouviu, me ensinou e se tornou um exemplo de cientista para mim, por sua seriedade, compaixão e amor ao conhecimento. Levarei para a vida as conversas e momentos de descontração dentro do laboratório junto a ele e aos colegas.

Seria injusto não agradecer meus colegas de laboratório, que tornaram-se amigos, especialmente o Matheus Colaço, pelo carinho e confiança, o Felipe Schneider, pelas conversas e dicas valiosas para a construção deste projeto, o Vinicius Glitz, pela compreensão sempre, o Vinícius Port, pelas piadas infames em momentos de tensão, e o Denner, pela amizade e descontração. Também agradeço à minha amiga Deise, por sempre me inspirar (sua companhia é um presente!). Agradeço também aos meus colegas de graduação que estiveram comigo nas disciplinas, compartilhando referências, estudando junto: Ana Paula, Morgana, Douglas, Nathalia, Hari. Vocês marcaram minha jornada positivamente, com muitas partilhas produtivas e diversão (pois precisamos sorrir para tornar a vida mais leve)!

Como menção honrosa, cito a Alexandra Elbakyan, pois sem a criação do SciHub, a bibliografia do referido documento estaria vazia. Além disso, coloco aqui meu respeito a todas às mulheres na ciência e tecnologia que construíram (e que ainda hão de construir) um caminho que me possibilita estar aqui e desenvolver este trabalho. É uma grande honra fazer parte dessa história.
\end{agradecimentos}
% ---

% ---
% Epígrafe
% ---
\begin{epigrafe}
	\vspace*{\fill}
	\begin{flushright}
		\textit{``
		Classification and theory are not ends in themselves. If they generate new experimental work, new compounds, new processes, new methods - they are good;
		if they are sterile - they are bad  \\
		(Bergmann, E. D, 1971)}
	\end{flushright}
\end{epigrafe}
% ---

% ---
% RESUMOS
% ---

% resumo em português
\setlength{\absparsep}{18pt} % ajusta o espaçamento dos parágrafos do resumo
\begin{resumo}
	\SingleSpacing
    Na química, é comum deparar-se com análises qualitativas de fenômenos através do uso de conceitos multidimensionais bastante difundidos, como a aromaticidade, por exemplo, descrita através de critérios geométricos, magnéticos e topológicos. De maneira geral e sucinta, é estabelecido que um composto aromático apresenta uma estabilização (entenda-se, decréscimo de energia e, consequentemente, de reatividade) devido à deslocalização eletrônica. Isso pode ser analisado de formas diversas, seja calculando a alternância dos comprimentos de ligação e/ou comparando as energias dos orbitais moleculares de fronteira de um composto aromático com um análogo sem deslocalização eletrônica. Nesse sentido, o presente trabalho mostra a ferramenta computacional produzida (chamada \textit{Balmy.jl}) com as linguagens de programação \gls{CSS}, \gls{HTML}, JavaScript e Julia para representar moléculas e seus orbitais tridimensionalmente, utilizando somente o navegador de \textit{internet} para realizar cálculos de índices \gls{HOMA}, \gls{HOMED} e \gls{EHMO}. A implementação dessas metodologias com interface gráfica acessível por um endereço eletrônico é uma novidade na química computacional, pois muitos \textit{softwares} já disponíveis exigem uma instalação exauriante para o usuário e não possuem uma interface gráfica responsiva e dinâmica. Os resultados apresentados demostram que os valores de \textit{gap} entre o \gls{HOMO} e o \gls{LUMO} estão de acordo com aqueles apresentados pela literatura. A aplicação \textit{web} produzida é, portanto, funcional e está bem implementada, podendo ser usada para análises rápidas da aromaticidade na pesquisa e como ferramenta didática de ensino, por ser gratuita, pública, de fácil uso e acesso.
	
	\textbf{Palavras-chave}: Aromaticidade. Multidimensional. \textit{Balmy.jl}. \gls{HOMA}. \gls{HOMED}. \gls{EHMO}
\end{resumo}

% resumo em inglês
\begin{resumo}[Abstract]
	\SingleSpacing
	\begin{otherlanguage*}{english}
In chemistry, it is common to come across qualitative analyses of phenomena through the use of multidimensional concepts, such as aromaticity, for example, described through geometric, magnetic, and topological criteria. In a general and succinct manner, it is established that an aromatic compound presents a stabilization (that is, a decrease in energy and, consequently, in reactivity) due to electron delocalization. This can be analyzed in various ways, either by calculating the alternation of bond lengths and/or comparing the energies of the frontier molecular orbitals of an aromatic compound with an analog without electron delocalization. In this sense, the present work shows the tool produced (called Balmy.jl) using the \gls{CSS}, \gls{HTML}, JavaScript, and Julia to represent molecules and their orbitals three-dimensionally, using only the web browser to \gls{HOMA}, and \gls{EHMO}. The presented results show good agreement with the values of \gls{HOMO}-\gls{LUMO} gaps presented in the literature. The electronic energies consistent results prove that the web application is functional and well implemented, and can be used for quick aromaticity analysis and as a didactic teaching tool.
		
		\textbf{Keywords}: Aromaticity. Multidimensional. Balmy.jl. \gls{HOMA}. \gls{HOMED}. \gls{EHMO}.
	\end{otherlanguage*}
\end{resumo}

%% resumo em francês 
%\begin{resumo}[Résumé]
% \begin{otherlanguage*}{french}
%    Il s'agit d'un résumé en français.
% 
%   \textbf{Mots-clés}: latex. abntex. publication de textes.
% \end{otherlanguage*}
%\end{resumo}
%
%% resumo em espanhol
%\begin{resumo}[Resumen]
% \begin{otherlanguage*}{spanish}
%   Este es el resumen en español.
%  
%   \textbf{Palabras clave}: latex. abntex. publicación de textos.
% \end{otherlanguage*}
%\end{resumo}
%% ---

{%hidelinks
	\hypersetup{hidelinks}
	% ---
	% inserir lista de ilustrações
	% ---
	\pdfbookmark[0]{\listfigurename}{lof}
	\listoffigures*
	\cleardoublepage
	% ---
	
	% ---
	% inserir lista de quadros
	% ---
	%%\pdfbookmark[0]{\listofquadrosname}{loq}
	%%\listofquadros*
	%%\cleardoublepage
	% ---
	
	% ---
	% inserir lista de tabelas
	% ---
	\pdfbookmark[0]{\listtablename}{lot}
	\listoftables*
	\cleardoublepage
	% ---
	
	% ---
	% inserir lista de abreviaturas e siglas (devem ser declarados no preambulo)
	% ---
	\imprimirlistadesiglas
	% ---
	
	% ---
	% inserir lista de símbolos (devem ser declarados no preambulo)
	% ---
	\imprimirlistadesimbolos
	% ---
	
	% ---
	% inserir o sumario
	% ---
	\pdfbookmark[0]{\contentsname}{toc}
	\tableofcontents*
	\cleardoublepage
	
}%hidelinks
% ---
% ---

% ----------------------------------------------------------
% ELEMENTOS TEXTUAIS
% ----------------------------------------------------------
\textual

% ---
% 1 - Introdução
% ---
% ----------------------------------------------------------
\chapter{Introdução}
% ----------------------------------------------------------

A profícua evolução do processamento computacional proporcionou avanços importantes em diversas áreas do conhecimento humano, como o \textit{design} de fármacos, o planejamento sintético e a ciência de materiais, com alto potencial de aplicabilidade. Essa tendência foi observada por Gordon E. Moore, químico estadunidense 
cofundador da \textit{Intel Corporation}. A partir dele, foi cunhada uma expressão com seu nome para designar o aumento binual de 100\% no número de trasistores dos chips microprocessadores, pelo mesmo custo. Isso possibilitou a implementação de ferramentas capazes de acessar - seja por meio de cálculos de estrutura eletrônica, 
simulações, ou predições - propriedades que não podem ser obtidas experimentalmente de forma direta. No entanto, usuários não especializados são, com frequência, desencorajados a usar tais instrumentos em função dos conceitos complexos e interfaces mal arquitetadas ou difíceis de lidar \autocite{Allouche2010}.

Ou seja, a computação gráfica auxilia na manipulação/representação direta dos objetos de estudo químico, sendo eles: átomos, moléculas (leia-se quaisquer agregados atômicos, independentemente da origem de suas interações) ou partes delas. Nesse seguimento, um dos avanços mais importantes é a aplicação da teoria de grafos à notação química e aos sistemas de busca de subestruturas e cálculos de propriedades, como a aromaticidade, que é muito sensível à geometria do sistema $\pi$, pois descreve as moléculas estabilizadas energeticamente pela deslocalização de elétrons móveis em ciclos (geometrias fechadas). Tal temática é extremamente explorada por trabalhos que vem sendo somados desde a primeira citação de Hoffmann na literatura, em 1855. Por exemplo, com uma busca sobre os termos aromático/aromaticidade no \textit{\href{https://scholar.google.com.br/scholar?hl=pt-BR&as_sdt=0\%2C5&as_ylo=2016&as_yhi=2022&q=aromatic&btnG=}{Google Scholar}} no período de 2016 a 2022, foram encontrados mais de 380 trabalhos/dia publicados, sendo a maioria destes na área de Química, uma vez que, entre os compostos carbocíclicos, destacam-se os derivados aromáticos, cuja estabilidade e reatividade dependem do caráter de deslocalização eletrônica. Para tais sistemas, é possível utilizar a equalização dos comprimentos de ligação como principal critério geométrico de análise quantitativa.


Ou seja, o presente trabalho pretende relacionar as propriedades extraídas das representações moleculares com a computação gráfica das estruturas químicas, produzindo uma interface gráfica para manipular esses compostos. A partir dessa ferramenta, o usuário será capaz de calcular os parâmetros de aromaticidade, cujo critério primário de identificação baseia-se no decréscimo de energia relativo ao efeito da conjugação eletrônica, denominada energia de estabilização aromática (do inglês, ASE). Por ser um conceito multidimensional, existe uma grande variedade de parâmetros capazes de avaliar quantitativamente o fator de estudo.

Além disso, a aromaticidade é um conceito explorado de forma superficial e pouco visual dentro das salas de aula. Uma vez que as dificuldades de compreensão por parte dos graduandos em relação aos conceitos e fenômenos originam-se na forma com que são apresentados\autocite{Cunha2018}, a ferramenta proposta poderá ser difundida para uso didático-pedagógico, podendo ser utilizada por discentes e docentes durante as aulas no sentido de demonstrar os modelos de classificação e a multidimensionalidade da aromaticidade, facilitando o entendimento.

% ----------------------------------------------------------
\section{Recomendações de uso}
% ----------------------------------------------------------

Este \emph{template} foi elaborado para se compilado em \LaTeX utilizando \abnTeX.  Todas as configurações de diferenciação gráfica nas divisões de seção e subseção seguem a  norma NBR 6027/2012 automaticamente. 

Uma nota de rodapé, já tem seu estilo automático com o comando \texttt{$\backslash$footnote}\footnote{As notas de rodapé possuem fonte tamanho 10. O alinhamento das linhas da nota de rodapé deve ser abaixo da primeira letra da primeira palavra da nota de modo dar destaque ao expoente.}.


% ----------------------------------------------------------
\section{Objetivos}
% ----------------------------------------------------------

Nas seções abaixo estão descritos o objetivo geral e os objetivos específicos deste TCC.

% ----------------------------------------------------------
\subsection{Objetivo Geral}
% ----------------------------------------------------------

Descrição...

% ----------------------------------------------------------
\subsection{Objetivos Específicos}
% ----------------------------------------------------------

Descrição...
% ---

% ---
% 2 - Capítulo 2
% ---
% ----------------------------------------------------------
\chapter{Metodologia
}\label{cap:desenvolvimento}
% ----------------------------------------------------------
\section{Interface gráfica}

De forma descritiva, o \textit{Balmy.jl} classifica-se como uma aplicação \textit{web}, isto é, um sistema projetado para utilização através de um navegador, lançando mão de linguagens \gls{HTML}, JavaScript, \gls{CSS} e Julia. Em geral, esse tipo de ferramenta é executada a partir de um servidor \gls{HTTP} ou localmente, no dispositivo do usuário. Na \gls{API} em questão (\textit{Balmy.jl}), a publicação foi feita no GitHub Pages, um servidor gratuito que hospeda sites criados a partir de repositórios Git. Esse servidor funciona como um sistema de controle de versionamento distribuído, aplicável principalmente ao desenvolvimento de \textit{software}, pois salva todo o histórico de edições contido no processo de construção de um programa computacional. A função do servidor \textit{web} é receber uma solicitação (requisição) e devolver (resposta) algo para o usuário, mediando essa interação que ocorre em tempo real. 

A infraestrutura da aplicação divide-se em duas partes. A primeira delas é o \textit{front-end}, que corresponde ao \textit{design} gráfico da interface com a qual o usuário lida. Esta é construída com JavaScript, \gls{HTML} e \gls{CSS}, de maneira composta, pois estas são as únicas linguagens de programação que o navegador consegue interpretar diretamente. Dentro do contexto, torna-se mandatório preocupar-se com a intuitividade dos elementos adicionados e como a experiência do usuário será delineada dentro da arquitetura da API. Do outro lado, temos o \textit{back-end}, que dá suporte para a interface e realiza as operações solicitadas sem que o usuário comunique-se diretamente com o código em Julia, linguagem criada especificamente para ciência e mineração de dados, álgebra linear (com a biblioteca nativa chamada \textit{LinearAlgebra.jl}) e aprendizado de máquina. Sua vantagem principal é a velocidade, devido ao fato de ser compilada via tradução dinâmica (\textit{Just In Time}) e por sua tipagem ser forte e dinâmica. Tais fatores contibuem para o incremento significativo no desempenho da linguagem Julia em relação à linguagem Python, por exemplo. No caso dos cálculos que serão realizados, a escolha é justificada porque a linguagem é otimizada para o uso matemático, devido à sua sintaxe adaptada para equações
e expressões numéricas.

As vantagens em desenvolver uma aplicação \textit{web}, cuja \gls{API} está hospedada em um servidor remoto, em relação a um aplicativo \textit{desktop} são muitas, a começar pela total exclusão da necessidade de instalação local, que torna-se muito complexa à medida que você adiciona dependências ou muda de sistema operacional. Em razão dessa escalabilidade, todas as atualizações do \textit{Balmy.jl} são/serão realizadas no servidor, ou seja, qualquer alteração na aplicação, já permite o acesso direto a todos os usuários de uma vez só. Como consequência, o custo de manutenção e o tempo de trabalho empregados para incrementar novas funcionalidades ao \textit{site} são reduzidos, pois tudo é feito em um servidor centralizado, sem que o usuário precise atualizar em seu computador pessoal.

Por fim, um último ponto a ser destacado é a usabilidade dos sistemas \textit{web}, que são pensados para oferecer uma experiência diferenciado ao usuário, baseada em múltiplos tipos de dispositivos. Um grande exemplo disso é a responsividade, que consiste em adaptar o conteúdo do sistema à tela do dispositivo utilizado. O endereço eletrônico do \textit{Balmy.jl}, a citar, pode ser acessado tanto por um computador, quanto por um dispositivo móvel (\textit{smartphone}), sendo altamente versátil.

%% vantagens


\section{Representação tridimensional}\label{desenhoestrutural}

As representações tridimensionais das moléculas foram feitas com o \textit{Three.js}, uma biblioteca que usa \textit{WebGL} para renderizar os objetos no espaço. No caso das moléculas, são aplicados modelos conhecidos, como o \textit{ball and stick}), utilizado para exibir geometrias em 3D por meio de bolas (átomos) e bastões (ligações químicas). As esferas são comumente coloridas de acordo com a convenção \gls{CPK}, que se popularizou por distinguir os elementos químicos com um bom contraste. Em 1952, Corey e Pauling publicaram uma descrição dos modelos de preenchimento de espaço de proteínas e outras biomoléculas que eles haviam construído na Caltech \autocite{Corey1953}. Tal proposição foi melhorada em 1965, por Koltun, que estendeu essa técnica aos halogênios e metais \autocite{Crossland2004-ll}. 

Várias das cores do \gls{CPK} referem-se mnemonicamente às cores dos elementos puros ou de seus compostos notáveis. Por exemplo, o hidrogênio é um gás incolor, o carbono como carvão vegetal ou grafite é preto, o enxofre em pó é amarelo, o cloro é um gás esverdeado, o bromo é um líquido vermelho escuro, o iodo no éter é violeta, o fósforo amorfo é vermelho, a ferrugem é vermelho alaranjado escuro, etc. Para algumas cores, tais como as de oxigênio e nitrogênio, a inspiração é menos clara. Talvez o vermelho para o oxigênio seja inspirado no fato de que o químico que contém oxigênio no sangue, hemoglobina, é vermelho vivo; e o azul para o nitrogênio é uma consequência dele ser o principal componente da atmosfera terrestre, que aos olhos humanos parece ser de cor azul (na tonalidade do céu).

Essa convenção foi muito provavelmente inspirada por modelos no século XIX. Em 1865, August Wilhelm von Hofmann, em uma palestra no \textit{Royal Institution} em Londres, estava usando modelos feitos de bolas de \textit{croquet} para ilustrar vários átomos (por exemplo, preto para o carbono, branco para o hidrogênio, verde para o cloro, vermelho para o oxigênio, azul para o nitrogênio)\autocite{Crossland2004-ll}. Na época, o \textit{croquet} era o esporte mais popular na Inglaterra, portanto, as bolas eram abundantes.

Para implementar esse modelo molecular utilizando o \textit{Three.js}, foi empregado o modelo de Blinn-Phong, capaz de simular superfícies brilhantes com holofotes especulares. O sombreamento é calculado \textit{per pixel} de acordo com o modelo de Phong. Desse modo, apesar de custar um pouco mais de tempo para renderizar, ele atinge uma acurácia maior do que outros modelos disponíveis no \textit{WebGL} (o de Lambert, por exemplo).

No modelo primeiramente proposto por Phong, é calculado continuamente o produto escalar $\gls{R} \cdot \gls{V}$ entre um observador $\gls{V}$ e o feixe de uma fonte de luz ($\gls{L}$) refletida ($\gls{R}$) sobre uma superfície. No ajuste proposto por Blinn, no entanto, calcula-se um vetor a meio caminho entre o espectador e os vetores de fonte luminosa, podendo substituir $\gls{R} \cdot \gls{V}$ por $\gls{N} \cdot \gls{H}$, onde $\gls{N}$ é o vetor normal à superfície.

Na \autoref{eq:R1}, $\gls{PH}$ é a matriz de Householder que reflete um ponto no hiperplano contendo a origem e um vetor normal $\gls{H}$. Este produto escalar é o cosseno de um ângulo que é metade do ângulo representado pelo produto escalar de Phong (se $\gls{V}$, $\gls{L}$, $\gls{N}$ e $\gls{R}$ estiverem todos no mesmo plano). Esta relação entre os ângulos sofre desvios discretos quando os vetores não se encontram no mesmo plano, especialmente em relação aos ângulos pequenos.

\begin{figure}[htb]
\begin{equation}
    \label{eq:R1}
    \gls{H} = \frac{\gls{L} + \highlight{blue}{$\gls{V}$} }{||\gls{L} + \tikzmarknode{value}{\highlight{blue}{$\gls{V}$}}||}
\end{equation}
\begin{tikzpicture}[overlay,remember picture,>=stealth,nodes={align=left,inner ysep=1pt},<-]
    \path (value.south) ++ (-1,-1.5em) node[anchor=north east,color=blue!67] (scalep){\textit{$\mathcal{V} = P_\mathcal{H} (-\mathcal{L})$}};
    \draw [color=blue!87](value.south) |- ([xshift=-1.65em,color=blue]scalep.north west);
\end{tikzpicture}
\vspace{2\baselineskip}
\end{figure}

Ainda sobre os ajustes de Blinn ao modelo de Phong, consideremos que o ângulo entre o vetor a meio caminho $(\gls{N} \cdot \gls{H})$ e a superfície normal é provavelmente menor do que o ângulo entre $\gls{R}$ e $\gls{V}$ usado no modelo original (a menos que a superfície seja vista de um ângulo muito íngreme para o qual é provável que seja maior), e já que Phong está usando $(\gls{R} \cdot \gls{V})^\mathfrak{a}$, um expoente (chamado de constante de brilho) pode ser definido como $\gls{cte}' > \gls{cte}$, tal que $(\gls{N} \cdot \gls{H})^{\mathfrak{a}'}$ está mais próxima da expressão original de Phong. Os resultados das estruturas geradas no \textit{Balmy.jl} estão na \autoref{design}.

Para detectar as ligações químicas, foi implementado um algoritmo baseado nos raios covalentes de van der Waals ($\gls{rw}$), que representa, pictoricamente, o raio de uma esfera sólida imaginária empregada para representar um átomo. Tal parâmetro é representado a partir de definições termodinâmicas da equação de estado de van der Waals (\autoref{vdw})\autocite{Volodina2022}.

\begin{equation}
\label{vdw}
    \gls{Vw} = \frac{4\pi}{3}  ({\gls{rw}})^3
\end{equation}

O parâmetro $\gls{Vw}$ é obtido experimentalmente e depende da natureza do gás. Ele é denominado volume de exclusão, referindo-se tanto ao volume próprio dos átomos, como ao volume circundante onde não pode haver outros porque nessa distância predominam as forças de repulsão entre os átomos do gás (forças de van der Waals). Desse modo, se a distância entre os dois átomos que estão sendo analisados é maior do que a soma dos seus raios de van der Waals, não é detectada uma ligação química pelo \textit{Balmy.jl}.

\section{Método de Hueckel Estendido}

O \gls{HMO} foi inicialmente proposto por Erich Hueckel em 1930 \autocite{Hckel1931} como uma alternativa para calcular orbitais moleculares como combinações lineares de orbitais atômicos\autocite{Coulson1978-ot}. Essa teoria prediz os orbitais moleculares de elétrons $\pi$ em moléculas com estrutura eletrônica deslocalizada, tal como o etileno, benzeno, butadieno e piridina. Isso nos fornece uma base teórica para fundamentar a regra de Hueckel, responsável por enunciar que moléculas cíclicas, planares ou espécies químicas iônicas com $4n + 2$ elétrons $\pi$ são aromáticos, sendo $n$ um número inteiro positivo $0, 1, 2, \cdots, n$.

Esse modelo baseia-se no fato de que a interação entre dois orbitais atômicos (sobreposição) produz um orbital molecular ligante mais estável (energia mais baixa), e um orbital molecular antiligante, menos estável do que aqueles que o geraram. Dessa forma, o número de orbitais moleculares novos é igual ao número de orbitais atômicos envolvidos (a combinação linear). A estabilidade relativa ou as energias dos orbitais moleculares em um polieno completamente conjugado, cíclico, planar pode ser predita pela teoria de Hueckel. Tomando o benzeno como exemplo, é possível ilustrar, através de um círculo de Frost (\autoref{fig:M1}), os níveis de energia dos seus orbitais moleculares de fronteira.

\begin{figure}[htb]
	\caption{\label{fig:M1} Círculo de Frost para o benzeno.}
	\begin{center}
		\includegraphics[width=0.60\textwidth]{images/figM.png}
	\end{center}
	\fonte{Autor(a).}
\end{figure}

%%% TODO : FROST DIAGRAM

Posteriormente, essa teoria foi estendida para tratar moléculas com heteroátomos (aqueles diferentes do carbono e hidrogênio)\autocite{Liwschitz1963}. Tal mudança foi feita por Roald Hoffmann \autocite{Hoffmann1963}, responsável por desenvolver o método do Orbital Molecular de Hueckel Estendido, do inglês \textit{Extended Hueckel Molecular
Orbital} (\gls{EHMO}). Esse método é classificado como semiempírico, pois utiliza dados experimentais para facilitar o processo de cálculo das integrais que acessam a interação elétron-elétron para contabilizar os efeitos de correlação eletrônica na estrutura molecular.

Como o objetivo do trabalho é desenvolver uma ferramenta rápida e eficiente para ser utilizada dentro no navegador de internet, esse método foi escolhido pelo seu baixo custo computacional de operação, devido às aproximações das matrizes Hamiltonianas pelo teorema de Koopman, cujas consequências fazem os valores de $H_{ii}$ serem substituídos pelas energias de ionização (leia o \autoref{ap:EHMO} para mais detalhes).

A metodologia de Hueckel estendido considera todos os elétrons da valência no cálculo (inclusive elétrons $\sigma$), e baseia-se no princípio de que os orbitais moleculares são combinações lineares de orbitais atômicos (assim como no método original de Hueckel). Ou seja, a reunião de todos os \gls{AOs} de valência presentes nos átomos do sistema em estudo, sobre os quais serão feitas iterações e se formarão os \gls{MOs}, é chamada \textit{conjunto de bases mínimo}. Como referência, estão implementados no \textit{Balmy.jl} os mesmos parâmetros empregados no \gls{YAeHMOP}\autocite{Avery2017}, que será usado para comparação dos tempos de execução e performance sobre um sistema em crescimento (um polímero). 

\begin{figure}[htb]
    \vspace{2\baselineskip}
\begin{equation}
    \label{eq:LCAO}
    \tikzmarknode{lcao}{\highlight{red}{LCAO}} \Longrightarrow \psi_j = \sum_{r=1}^{N} c_{jr} \tikzmarknode{aos}{\highlight{blue}{$\phi_r$}}
\end{equation}
\begin{tikzpicture}[overlay,remember picture,>=stealth,nodes={align=left,inner ysep=1pt},<-]
    \path (lcao.north) ++ (-0.70,1.5em) node[anchor=south east,color=red!67] (scalep){\textit{combinação linear}};
    \draw [color=red!87](lcao.north) |- ([xshift=-0.70em,color=red]scalep.south west);
    
    \path (aos.south) ++ (1,-1.5em) node[anchor=north west,color=blue!67] (scalep){\textit{orbitais atômicos}};
    \draw [color=blue!87](aos.south) |- ([xshift=1.65em,color=blue]scalep.north east);
\end{tikzpicture}
\vspace{2\baselineskip}
\end{figure}

A energia do \textit{j}-ésimo orbital é dada pela equação monoeletrônica de Schroedinger ($\gls{Hamilt}_{eff} \psi_j = \epsilon_j \gls{wave}_j$) usando um Hamiltoniano ($\hat{H}_{eff}$), que expressa a interação de um elétron com o restante da molécula.

\begin{figure}[htb]
    \vspace{3\baselineskip}
\begin{equation}
\label{eq:energy_j1}
        \epsilon_j = \frac{\tikzmarknode{hamil}{\highlight{blue}{$\displaystyle \langle \psi_j | \hat{H}_{eff} | \psi_j \rangle$}}}{\tikzmarknode{overlap}{\highlight{red}{$\displaystyle \langle \psi_j | \psi_j \rangle$}}}
\end{equation}
\begin{tikzpicture}[overlay,remember picture,>=stealth,nodes={align=left,inner ysep=1pt},<-]
    \path (hamil.north) ++ (-0.70,1.5em) node[anchor=south east,color=blue!67] (scalep){\textit{$\displaystyle \int \psi_j  \times \hat{H}_{eff} \times \psi_j d\tau$}};
    \draw [color=blue!87](hamil.north) |- ([xshift=-0.70em,color=blue]scalep.south west);
    
    \path (overlap.south) ++ (1,-1.5em) node[anchor=north west,color=red!67] (scalep){\textit{$\displaystyle \int \psi_j \times \psi_j d\tau$}};
    \draw [color=red!87](overlap.south) |- ([xshift=1.65em,color=red]scalep.north east);
\end{tikzpicture}
\vspace{2\baselineskip}
\end{figure}

A \autoref{eq:energy_j1} pode ser escrita como a \autoref{eq:energy_j} ao substituirmos a \autoref{eq:LCAO} na \autoref{eq:energy_j}.

\begin{equation}
\label{eq:energy_j}
    \epsilon_j = \frac{ \displaystyle \bigg{\langle} \sum_{r=1}^{N} c_{jr} \psi_r \bigg{|} \hat{H}_{eff} \bigg{|} \sum_{s=1}^{N} c_{js} \psi_s \bigg{\rangle}}{\displaystyle \bigg{\langle} \sum_{r=1}^{N} c_{jr} \psi_r \bigg{|} \sum_{s=1}^{N} c_{js} \psi_s \bigg{\rangle}}
\end{equation}

Generalizando a expressão da \autoref{eq:energy_j} para quaisquer orbitais moleculares, é omitido o índice $j$ (\autoref{eq:energy}).

\begin{equation}
\label{eq:energy}
    \epsilon = \frac{\displaystyle \sum_{r=1}^{N} \sum_{s=1}^{N} c^*_r c_s \langle \psi_r | \hat{H}_{eff} | \psi_s \rangle}{\displaystyle \sum_{r=1}^{N} \sum_{s=1}^{N} c^*_r c_s \langle \psi_r | \psi_s \rangle}
\end{equation}

Da \autoref{eq:energy} decorrem as integrais moleculares a serem calculadas (para uma melhor compreensão teórica e detalhamento matemático dessa seção, leia o \autoref{ap:HMO} e o \autoref{ap:EHMO}):

\begin{itemize}
    \item $\displaystyle S_{rs} = \langle \psi_r | \psi_j \rangle$ é a integral de sobreposição. Como estamos trabalhando com orbitais normalizados, $S_{rr} = 1$;
    
    \item $\displaystyle H_{rr} = \langle \psi_r | \hat{H}_{eff} | \psi_r \rangle$ é a integral de Coulomb, aproximada pelo oposto do potencial de ionização (como é descrito no \autoref{ap:EHMO});
    
    \item $\displaystyle H_{rs} = \langle \psi_r | \hat{H}_{eff} | \psi_s \rangle$ é a integral de ressonância. Essa integral nos dá a energia de um elétron na região do espaço onde as funções $\phi_r$ e, $\phi_s$ sobrepõem-se. Os valores de $H_{rs}$ são computados pela \autoref{hamiltonian} (no \autoref{ap:EHMO}).
\end{itemize}

Substituindo então os elementos das integrais supracitadas na \autoref{eq:energy}, obtemos a \autoref{eq:energy_cont}.

\begin{equation}
\label{eq:energy_cont}
    \epsilon = \frac{\displaystyle \sum_{r=1}^{N} \sum_{s=1}^{N} c^*_r c_s H_{rs}}{\displaystyle \sum_{r=1}^{N} \sum_{s=1}^{N} c^*_r c_s S_{rs}}
\end{equation}

\subsection{Integral de sobreposição}\label{sec:overlap}

No Método de Hueckel Estendido, as integrais de sobreposição são calculadas utilizando \gls{STO} (representado por coordenadas esféricas na \autoref{stos}), pois seus resultados são mais acurados (mesmo que não sejam solução exata), principalmente pelo fato de reproduzirem mais precisamente a região do cúspide e o decaimento orbital. Essa aproximação é feita porque a descrição matemática dos orbitais atômicos não pode ser exatamente conhecida, mas aproximada pela forma dos orbitais dos sistemas hidrogeniônicos, uma vez que o problema torna-se intratável analiticamente para qualquer sistema multieletrônico. Desse modo, os conjuntos de bases são obtidos pelas regras de Slater\autocite{Slater1930, Lu2006}.

\begin{figure}[htb]
    \vspace{3\baselineskip}
\begin{equation}
\label{stos}
        \tikzmarknode{sto}{\highlight{red}{$\chi_{nlm}(\zeta, r)$}} = \mathbf{N} \cdot r^{n-1} e^{-\zeta r} \cdot \tikzmarknode{harmonic}{\highlight{blue}{$Y_l^m (\theta, \Phi)$}}
\end{equation}
\begin{tikzpicture}[overlay,remember picture,>=stealth,nodes={align=left,inner ysep=1pt},<-]
    \path (sto.south) ++ (-0.70,-1.5em) node[anchor=north east,color=red!67] (scalep){\textit{Slater Type Orbital}};
    \draw [color=red!87](sto.south) |- ([xshift=-0.70em,color=red]scalep.north west);
    
    \path (harmonic.north) ++ (2.00,1.5em) node[anchor=south west,color=blue!67] (scalep){\textit{função angular}};
    \draw [color=blue!87](harmonic.north) |- ([xshift=2.00em,color=blue]scalep.south east);

\end{tikzpicture}
\vspace{2\baselineskip}
\end{figure}

\noindent onde:

\begin{equation}
\label{normalizeee}
   \mathbf{N} = \sqrt{\frac{(2 \zeta)^{n-1}}{(2n)!}} 
\end{equation}

\begin{equation}
\label{angular3}
    \textcolor{blue!87}{Y_l^m (\theta, \Phi) = i^{m + |m|}\bigg{[}\displaystyle \frac{(2l+1)(l+|m|)!}{2(l + |m|)!}\bigg{]}^{1/2} P^m_l (cos \; \theta) \phi_m(\Phi)}
\end{equation}

Avaliando a \autoref{stos}, \gls{normalize} (\autoref{normalizeee}) representa a constante de normalização, $r^{n-1}e^{-\zeta r}$ representa a parte radial da função (onde $r$ é a distância entre o elétron e o núcleo do átomo e $\zeta$ é uma constante de proteção relacionada à carga efetiva do núcleo, sendo a carga nuclear parcialmente protegida por elétrons), e $Y_l^m (\theta, \Phi)$ representa a parte angular, onde os esféricos harmônicos normalizados são relacionados com polinômios de Legendre, definidos pela fórmula geral na \autoref{legendre}, e $\phi_m (\Phi)$ representa as funções ortonormais.

\begin{equation}
    \phi_m (\Phi) = (2\pi)^{-1/2} e^{im\Phi}
\end{equation}


\begin{equation}
\label{legendre}
    P^m_l(x) = (1 - x^2)^{m/2} \sum_{u=0}^{l - m} \highlight{blue}{$C_{lmu}$} x^u
\end{equation}

\noindent onde:

\begin{equation}
    \textcolor{blue!87}{C_{lmu} = \displaystyle \frac{(-1)^{(l-m-u)/2}[1 + (-1)^{(l - m - u)}](l + m + u)!}{2^{l + 1}([l - m - u]/2)!([l + m + u]/2)!}}
\end{equation}

Decorre, portanto, a expressão das integrais de sobreposição \autocite{Hoggan2011}
definida por coordenadas esféricas.

\begin{equation}
\label{overlap}
    S^{n_2 l_2 m_2}_{n_1 l_1 m_1} (\zeta_1; \zeta_2; \textbf{\textit{R}} ) = \int [\chi_{n_1 l_1 m_1} (\zeta_1, \textbf{\textit{r}})]^* \chi_{n_2 l_2 m_2} (\zeta_2, \textbf{\textit{r - R}})
\end{equation}

A expressão da \autoref{overlap} pode ser aproximada a uma expressão analítica por uma série de somatórios finitos ponderados por coeficientes binomiais Isso torna-as mais facilmente computáveis e, portanto, calculáveis.

\begin{figure}[htb]
\begin{equation}
\label{analitica}
\begin{split}
    S^{n_2 l_2 m_2}_{n_1 l_1 m_1} (\zeta_1; \zeta_2; \textbf{\textit{R}} ) = \mathbf{N} \mathbf{N'} (R/2)^{n_1 + n_2 + 1} \highlight{blue}{$D(l_1, l_2, m)$}  \\[0.35cm] \sum_{u=0}^{l_1 - m} \sum_{v=0}^{l_2 - m} \sum_{p=0}^{n_1 - m - u} \sum_{p'=0}^{n_2 - m - v} \sum_{q=0}^{m} \sum_{q'=0}^{m} \sum_{t=0}^{u}  \sum_{t'=0}^{v} (-1)^{n_1 - m - u} C_{l_1 m u} C_{l_2 m v} \\[0.35cm] F_p (n_1 - m - u) F_{p'} (n_2 - m - v)
    F_q (m) F_{q'} (m) F_t (u) F_{t'} (v) A_i (\alpha) B_j (\beta) 
\end{split}
\end{equation}
\end{figure}

\noindent onde:

\begin{equation}
    \textcolor{blue!87}{D(l_1, l_2, m) = \displaystyle \sqrt{\frac{(2l_1 + 1)(2l_2 + 1)(l_1 - m)!(l_2 - m)!}{4(l_1 + m)!(l_1 + m)!}}}
\end{equation}

Nessa expressão (\autoref{analitica}), $A_i (\alpha)$ e $B_j (\beta)$ são funções auxiliares, $\alpha = R(\zeta_1 + \zeta_2)/2$, $\beta = R(\zeta_1 - \zeta_2)/2$ e $F_n(m)$ são os coeficientes binomiais \autocite{Mekelleche1997} e os termos $i$ e $j$ estão descritos na \autoref{termos} e na Equação (27).

\begin{align}
\label{termos}
    i = n_1 + n_2 - 2m - p - p' + 2q - t - t' \\[0.35cm]
    j = p + p' + 2m - 2q' + u + v - t - t'
\end{align}

%\section{Otimização de geometria}

%Predizer o arranjo mais estável dos átomos em uma molécula é uma das mais importantes tarefas na %química quântica computacional. Essencialmente, este é um problema de otimização onde a energia total %da molécula é minimizada com relação às posições dos núcleos atômicos. A geometria molecular obtida %desse cálculo é, dessa forma, um ponto de partida para inúmeras simulações de propriedades %moleculares. Se a geometria não é acurada, então quaisquer cálculos que derivam dele também podem ser %espúrios.

%Uma vez que os núcleos são muito mais pesados do que os elétrons, nós podemos tratá-los como %partículas pontuais associadas às suas respectivas posições. A partir disso, é possível afirmar que a %energia da molécula $E(x)$ depende das coordenadas nucleares $x$, as quais definem a energia de %superfície potencial. Resolver, portanto, o problema estacionário $\nabla_x E(x)$, corresponde à %otimização das coordenadas nucleares, e a partir delas é possível determinar a energia de equilíbrio %da molécula.

%No trabalho em questão, foi utilizado um método genérico (variacional) para encontrar a estrutura %situada na região de mínima energia. A ideia central desse algoritmo é considerar explicitamente o %Hamiltoniano $H(x)$ como uma observável parametrizada, que depende das coordenadas nucleares $x$. O %objetivo desse procedimento é encontrar o mínimo global da função de custo global $g(\theta, x) = < %\Psi(\theta) | H(x) | \Psi(\theta) >$.



\section{Teoria de grafos}

Ao acessar a interface gráfica, será possível ao usuário fornecer um arquivo de extensão \textit{.xyz}, contendo as informações em coordenadas cartesianas da estrutura a ser analisada (\autoref{fig:input}). O sistema de interesse será processado computacionalmente e transformado em um grafo $G$, que corresponde a uma coleção de vértices (pontos) chamados genericamente de $V$ e arestas (linhas) denotadas por $E$. Formalmente um grafo simples $G$ é definido como um par ordenado $(V(G), E(G))$, o qual consiste de um conjunto $V(G)$ de vértices $V$ não vazio e um conjunto de arestas $E(G) = E$ contendo pares não ordenados de elementos distintos de V, uma vez que cada elemento de $E(G)$ é uma linha que conecta dois pontos de $V(G)$. Para maiores detalhes sobre a teoria de grafos, acesse o \autoref{ap:graph}.

\begin{figure}[htb]
	\caption{\label{fig:input} Exemplo de um arquivo de \textit{input} (\textit{.xyz}) para o \textit{Balmy.jl}. Essa geometria molecular de partida foi otimizada a um nível de teoria CCSD(T), com uma base def2-TZVP, e correção de dispersão de Grimme D4.}
	\begin{center}
		\includegraphics[width=0.8\textwidth]{images/fig2(4).png}
	\end{center}
	\fonte{Autor(a)}
\end{figure}



Desse modo, transpõe-se a representação apresentada na \autoref{fig:M2} para as estruturas químicas de interesse, uma vez que os átomos podem ser considerados os análogos químicos dos vértices, e as ligações, das arestas. Enumerando-se sequencialmente os nodos do grafo derivado do benzeno como na \autoref{fig:graphEnumerated}, é possível então verificar computacionalmente onde esses pontos estão localizados. Ao identificar a estrutura como um grafo, é mais fácil construir a matriz do determinante secular da teoria \gls{EHMO} (veja a discussão desse resultado na \autoref{sec:benzene}). 

\newpage

\begin{figure}[htb]
	\caption{\label{fig:M2} Exemplo ilustrativo de um grafo cíclico não direcionado. Os pontos em azul representam os seis vértices que se conectam através das linhas pretas correspondentes às arestas. O equivalente à esquerda é um anel benzênico de Kekulé, com seis átomos de carbono ocupando os nodos de um ciclo hexagonal.}
	\begin{center}
		\includegraphics[width=0.55\textwidth]{images/grafo(2).png}
	\end{center}
	\fonte{Autor(a).}
\end{figure}

\begin{figure}[htb]
\caption{\label{fig:graphEnumerated}Representação do grafo mostrado na \autoref{fig:M2} com nodos enumerados sequencialmente de 1-6.}
	\begin{center}
		\includegraphics[width=0.55\textwidth]{images/graphEnumerated.png}
	\end{center}
	\fonte{Autor(a)}
\end{figure}

Além desse uso, o grafo também será aplicado para o cálculo dos índices geométricos (\gls{HOMA}, \gls{rHOMA}, \gls{HOMED}) implementados no \textit{Balmy.jl}. No caso das estruturas acíclicas, não há maiores dificuldades, mas para estruturas cíclicas, como o próprio benzeno mostrado na \autoref{fig:graphEnumerated}, é necessário implementar um tipo de algoritmo específico que identifique o anel (seja ele aromático ou não) da estrutura.

O \gls{DFS}\autocite{Knuth1997-jf, Goodrich2001-pd} é um algoritmo recursivo que perpassa todos os vértices de um grafo ou de uma árvore de dados através do conceito de \textit{backtracing} (retorno). Ou seja, ele começa em um nodo raiz definido arbitrariamente e a partir dele explora suas adjacências através da expansão da árvore de busca, aprofundando-se até que o alvo da busca seja encontrado ou até que ele se depare com um nó que não possui adjacências (nodo folha). Então a busca retrocede (\textit{backtrack}) e começa no próximo nó. Numa implementação não-recursiva, todos os nós expandidos recentemente são adicionados a uma pilha, para realizar a exploração (Algoritmo \ref{alg:1}, \autoref{fig:DFS}).




\begin{figure}[htb]
\caption{\label{fig:DFS} Representação esquemática do algoritmo DFS (Algoritmo \ref{alg:1}). Todos os nodos adjacentes são visitados até que sejam marcados como visitados (cinza). O ciclo é encontrado quando o último nodo é igual ao nodo raiz.}
	\begin{center}
		\includegraphics[width=0.65\textwidth]{images/DFS.png}
	\end{center}
	\fonte{Autor(a)}
\end{figure}

\SetKwComment{Comment}{/* }{ */}

\begin{algorithm}
\caption{Detecção de ciclos em grafos via DFS}\label{alg:1}
\LinesNumbered
\KwData{vértice geral $v_n$ }
\KwResult{$true$ se o ciclo é encontrado}
\SetKwFunction{FMain}{detectcycle}

  \SetKwProg{Pn}{Function}{}{}
  \Pn{\FMain{$v_n$}}{$mark(v_n, visited\;)$\;
        \For{$ v_{n'} \in \; neighbors(v_n) $}{
        \eIf{$mark(v_n) == \; visited$}
        {\If{$v_n ==v_{n'}$ \textbf{or} $v_n \; != \; parent(v_{n'})$}{\textbf{return} $true$ \;}}
        {\If{$detectcycle(v_{n'})$}{\textbf{return $true$} \;}}
    }
    }
\end{algorithm}

Desse modo, ao identificar os ciclos a partir do algoritmo \gls{DFS}, é possível classificar os ciclos, análogos aos anéis aromáticos, e calcular os índices geométricos (para uma descrição do formalismo dos índices geométricos, veja a \autoref{sec:HOMA}) referente a cada um dos sextetos das estruturas policíclicas (veja um exemplo dessa aplicação na \autoref{sec:policicle}). 

\newpage

\section{Tratamento de resíduos}

Como o trabalho em questão é teórico, foram utilizadas ferramentas computacionais dentro do Grupo de Estrutura Eletrônica Molecular (GEEM) da Universidade Federal de Santa Catarina (UFSC) para o desenvolvimento do mesmo. Desse modo, em caso de geração de lixo eletrônico, o descarte foi feito de forma apropriada junto às estações de coleta seletiva específicas, conhecidas como Ecopontos.
% ---

% ---
% 3 - Capítulo 3
% ---
\include{chapters/3-chapter}
% ---

% ---
% 4 - Resultados
% ---
%\phantompart
% ----------------------------------------------------------
\chapter{Resultados}
% ----------------------------------------------------------

\section{Interface gráfica}\label{design}

A interface gráfica do \textit{Balmy.jl} é mostrada na \autoref{gui-pronta}. Os elementos interativos foram implementados para garantir o suporte e usabilidade do \textit{software}, sendo extremamente intuitivo e fácil de usar, pois é acessado através do navegador e, assim, pode utilizado nas salas de aula da graduação.

\begin{figure}[htb]
	\caption{\label{workflow} Fluxo de trabalho do \textit{software} produzido.}
	\begin{center}
		\includegraphics[width=0.6\textwidth]{images/workflow.png}
	\end{center}
	\fonte{Autor(a)}
\end{figure}

\begin{figure}[htb]
	\caption{\label{gui-pronta} Aqui é mostrada a interface gráfica produzida. Os círculos sinalizam as funções associadas às áreas da \gls{GUI}.}
	\begin{center}
		\includegraphics[width=0.8\textwidth]{images/GUI-EXAMPLE.png}
	\end{center}
	\fonte{Autor(a)}
\end{figure}

O fluxo de trabalho básico de funcionamento do \textit{Balmy.jl} também é ilustrado pelo esquema da \autoref{workflow}, onde é possível notar que o usuário insere, como entrada, o arquivo contendo as coordenadas cartesianas do sistema químico que deseja estudar. A \autoref{fig:input} mostra um exemplo de \textit{input} do \textit{Balmy.jl} para a molécula de benzeno. É um arquivo de extensão \textit{.xyz} que contém as coordenadas cartesianas da molécula de interesse. Na primeira linha, temos o nome da estrutura, em seguida, um comentário, e a partir da terceira linha são mostradas as informações da geometria em quatro colunas. A primeira delas contém os símbolos dos átomos contidos na molécula. Da segunda à quarta coluna, são indicadas as coordenadas cartesianas do eixo $x$, $y$ e $z$, respectivamente. Esses dados aparecem na caixa de texto no lado direito da tela (espaço indicado na \autoref{gui-pronta}).

\begin{figure}[htb]
	\caption{\label{fig:input} Exemplo de um arquivo de \textit{input} (\textit{.xyz}) para o \textit{Balmy.jl}.}
	\begin{center}
		\includegraphics[width=0.8\textwidth]{images/fig2(4).png}
	\end{center}
	\fonte{Autor(a)}
\end{figure}

A partir das informações geométricas fornecidas pelo usuário, a estrutura é renderizada e representada tridimensionalmente pelo modelo das superfícies de Blinn-Phong (veja a \autoref{desenhoestrutural} para uma melhor descrição desse método). Nesse sentido, o \textit{Balmy.jl} é bastante versátil, pois é possível customizar o modelo através dos efeitos de iluminação empregados nas estruturas. Por exemplo, variando o valor de $\alpha'$ (coeficiente de brilho), é possível notar que, na \autoref{fig:representations}, quanto maior for o valor, menor será o holofote especular, representado pelos pontos luminosos nas estruturas moleculares.

%%% discutir equação do alpha

\begin{figure}[htb]
\caption{\label{fig:representations} Exemplo da aplicação do modelo de Blinn-Phong na molécula de benzenos a partir da alteração dos coeficientes $\alpha'$ para os átomos (representados por esferas). Da direita para a esquerda, os valores atribuídos foram: 0, 10, 50, 100, 1000; respectivamente.}
	\begin{center}
		\includegraphics[width=1.0\textwidth]{images/shininess(1).png}
	\end{center}
	\fonte{Autor(a)}
\end{figure}

Antes da realização dos cálculos, é possível ainda acessar a terceira aba do \textit{Balmy.jl} para modificar o padrão das configurações (\autoref{fig:conf}). Por exemplo, é possível alterar a carga do composto (embora, por definição, o \textit{Balmy.jl} considere a moléula neutra) e o parâmetro de Wolfsberg-Helmholtz ($K$) (o padrão adotado é 1.75, parametrizado para hidrocarbonetos \autocite{Hoffmann1963}), que será discutido mais profundamente na \autoref{wolfsberg}.

\begin{figure}[htb]
	\caption{\label{fig:conf} Aqui, é mostrada a aba de configurações do cálculo realizado no \textit{Balmy.jl}}
	\begin{center}
		\includegraphics[width=0.8\textwidth]{images/conf.png}
	\end{center}
	\fonte{Autor(a)}
\end{figure}


Ao clicar no botão \textit{Start}, o usuário emite a requisição ao servidor para que o cálculo de \gls{EHMO} seja iniciado. Em questão de segundos, será possível acessar os resultados relativos ãs funções de onda e aos \gls{MOs} na segunda aba (\autoref{fig:results}). Para o caso do benzeno, os resultados serão discutidos na \autoref{sec:benzene}.

\begin{figure}[htb]
	\caption{\label{fig:results} Aqui, é mostrada a aba de resultados do cálculo realizado no \textit{Balmy.jl}}
	\begin{center}
		\includegraphics[width=0.8\textwidth]{images/results.png}
	\end{center}
	\fonte{Autor(a)}
\end{figure}

\section{Diagramas de orbitais moleculares}\label{sec:benzene}

Como o enfoque desse trabalho é a análise de compostos aromáticos, analisaremos de maneira mais aprofundada o exemplo do benzeno, cujas coordenadas estão mostradas na \autoref{tab:coords}. Os comprimentos de ligação obtidos para \ce{C-C} são $1.40$ \AA, e para \ce{C-H} são $1.10$ \AA.

\begin{table}[htb]
	\centering
	\caption{\label{tab:coords} Coordenadas atômicas do benzeno.}	
	\begin{tabular}{crrr}
		\toprule
		\textbf{Átomo (posição)} & coordenada $x$ & coordenada $y$ & coordenada $z$
		\\ 
		\midrule
C(1)  &    0.000000000000  &   1.950000000000  &   1.391500000000  \\
C(2)  &    1.205074349366  &   1.950000000000  &   0.695750000000  \\
C(3)  &    1.205074349366  &   1.950000000000  &  -0.695750000000  \\
C(4)  &   -0.000000000000  &   1.950000000000  &  -1.391500000000  \\
C(5)  &   -1.205074349366  &   1.950000000000  &  -0.695750000000  \\
C(6)  &   -1.205074349366  &   1.950000000000  &   0.695750000000  \\
H(1)  &    0.000000000000  &   1.950000000000  &   2.471500000000  \\
H(2)  &    2.140381785453  &   1.950000000000  &   1.235750000000  \\
H(3)  &    2.140381785453  &   1.950000000000  &  -1.235750000000  \\
H(4)  &   -0.000000000000  &   1.950000000000  &  -2.471500000000  \\
H(5)  &   -2.140381785453  &   1.950000000000  &  -1.235750000000  \\
H(6)  &   -2.140381785453  &   1.950000000000  &   1.235750000000  \\
    \bottomrule
	\end{tabular}
	\fonte{Autor(a).}
\end{table}

Seguindo o grafo mostrado na \autoref{fig:M2} e enumerado na \autoref{fig:graphEnumerated}, é possível construir uma matriz de adjacência de dimensão $n \times n$, onde $n$ é o número de átomos, excluindo-se os hidrogênios. No caso do benzeno, a ordem da matriz em questão é 6.

\begin{figure}[htb]
\vspace{0.8\baselineskip}
\begin{equation}
\label{eq:adjmatrix}
\begin{bmatrix}
    0 & 1 & 0 & 0 & 0 & 1 \\
    1 & 0 & 1 & 0 & 0 & 0 \\
    0 & 1 & 0 & 1 & 0 & 0 \\
    0 & 0 & 1 & 0 & 1 & 0 \\
    0 & 0 & 0 & 1 & 0 & 1 \\
    1 & 0 & 0 & 0 & 1 & 0
\end{bmatrix}
\end{equation}
\end{figure}

\noindent Consideremos $a_{ij}$ o elemento geral associado à matriz da \autoref{eq:5}, sendo igual a uma unidade quando os carbonos $i$ e $j$ estão ligados. Analogamente, na teoria de Hueckel e de Hueckel estendido, a matriz do determinante secular (\autoref{eq:secularmatrix}) é aquela cujos autovalores determinam as energias dos orbitais moleculares.

\begin{figure}[htb]
\vspace{0.8\baselineskip}
\begin{equation}
\label{eq:secularmatrix}
\begin{bmatrix}
    \alpha - E & \beta & 0 & 0 & 0 & \beta \\
    \beta & \alpha - E & \beta & 0 & 0 & 0 \\
    0 & \beta & \alpha - E & \beta & 0 & 0 \\
    0 & 0 & \beta & \alpha - E & \beta & 0 \\
    0 & 0 & 0 & \beta & \alpha - E & \beta \\
    \beta & 0 & 0 & 0 & \beta & \alpha - E \\
\end{bmatrix}
\end{equation}
\end{figure}

No caso do benzeno, ao resolver o determinante secular da matriz anterior (através do procedimento mostrado no \autoref{ap:HMO} e no \autoref{ap:EHMO}), obtemos a equação

\begin{figure}[htb]
\begin{equation}
    \label{eq:R3}
    \tikzmarknode{variable}{\highlight{blue}{$x$}}^6 - 6\highlight{blue}{$x$}^4 + 9\highlight{blue}{$x$}^2 = 0
\end{equation}
\begin{tikzpicture}[overlay,remember picture,>=stealth,nodes={align=left,inner ysep=1pt},<-]
    \path (variable.south) ++ (1,-1.5em) node[anchor=north west,color=blue!67] (scalep){\textit{$x = \displaystyle \frac{\alpha - E}{\beta}$}};
    \draw [color=blue!87](variable.south) |- ([xshift=1.65em,color=blue]scalep.north east);
\end{tikzpicture}
\vspace{2\baselineskip}
\end{figure}

\begin{figure}[htb]
\vspace{2\baselineskip}
\begin{equation}
    \label{eq:orb_mols}
\begin{split}
    \tikzmarknode{e1}{\highlight{blue}{$E_1$}} = \alpha +  2 \beta \\[0.3cm]
    \tikzmarknode{e2}{\highlight{red}{$E_2 = E_3$}} = \alpha + \beta \\
    \tikzmarknode{e3}{\highlight{red}{$E_4 = E_5$}} = \alpha - \beta \\[0.3cm]
    \tikzmarknode{e6}{\highlight{blue}{$E_6$}} = \alpha - 2\beta
\end{split}
\end{equation}
\begin{tikzpicture}[overlay,remember picture,>=stealth,nodes={align=left,inner ysep=1pt},<-]
    \path (e1.north) ++ (1,1.5em) node[anchor=south west,color=blue!67] (scalep){\textit{HOMO-1}};
    \draw [color=blue!87](e1.north) |- ([xshift=1.65em,color=blue]scalep.south east);

    \path (e2.north) ++ (-0.60,0.6em) node[anchor=south east,color=red!67] (scalep){\textit{HOMO (orbitais degenerados)}};
    \draw [color=red!87](e2.north) |- ([xshift=-1.65em,color=red]scalep.south west);
    
    \path (e3.south) ++ (-0.60,-0.6em) node[anchor=north east,color=red!67] (scalep){\textit{LUMO (orbitais degenerados)}};
    \draw [color=red!87](e3.south) |- ([xshift=-1.65em,color=red]scalep.north west);
    
    \path (e6.south) ++ (1,-1.5em) node[anchor=north west,color=blue!67] (scalep){\textit{LUMO+1}};
    \draw [color=blue!87](e6.south) |- ([xshift=1.65em,color=blue]scalep.north east);
\end{tikzpicture}
\end{figure}

Resolvendo o polinômio da \autoref{eq:R3}, conseguimos suas raízes ($x = \pm 1, \pm 1, \pm 2$) e, portanto, as energias para seis orbitais moleculares do benzeno (HOMO-1, HOMO, LUMO, LUMO+1). Combinando as raízes com os parâmetros $\alpha$ e $\beta$, nós temos a \autoref{eq:orb_mols} (\autoref{fig:MOs}).

\begin{figure}[htb]
\caption{\label{fig:MOs} Diagrama de energia dos orbitais moleculares do benzeno.}
	\begin{center}
		\includegraphics[width=0.75\textwidth]{images/MOs.png}
	\end{center}
	\fonte{Autor(a).}
\end{figure}

Os autovalores repetidos vão gerar orbitais degenerados, isto é, aqueles que estão no mesmo nível energético, como é possível notar no diagrama e nos valores de energia obtidos. Esses mesmos valores podem ser obtidos se calcularmos os autovalores da matriz de adjacência. Desse modo, comprovamos que, ao solucionar o determinante e os autovalores da matriz secular, é possível obter os valores de energia e as populações associadas a cada orbital molecular da molécula de interesse.

O diagrama de níveis de energia para o benzeno é dado na \autoref{fig:MOs}. Os seis elétrons $\pi$ são colocados nos três níveis de energia mais baixos. Desse modo, a energia eletrônica em benzeno é definida pela \autoref{epi}.

\begin{equation}
\label{epi}
    E_\pi = 2(\alpha + 2\beta) + 4(\alpha + \beta) = 6\alpha + 8\beta
\end{equation}

As funções de onda resultantes para os seis orbitais moleculares $\pi$ do benzeno são dados pela \autoref{wavefunctions}.

\begin{equation}
\label{wavefunctions}
\begin{split}
    \psi_1 = \frac{1}{\sqrt{6}}(2p_{z1} + 2p_{z2} + 2p_{z3} + 2p_{z4} + 2p_{z5} + 2p_{z6}) \Longrightarrow E_1 = \alpha + 2\beta \\
    \psi_2 = \frac{1}{\sqrt{4}}(2p_{z2} + 2p_{z3} - 2p_{z5} - 2p_{z6}) \Longrightarrow E_2 = \alpha + \beta \\
   \psi_3 = \frac{1}{\sqrt{3}}(2p_{z1} + \frac{1}{2} 2p_{z2} - \frac{1}{2} 2p_{z3} - 2p_{z4} - \frac{1}{2} 2p_{z5} + \frac{1}{2} 2p_{z6}) \Longrightarrow E_3 = \alpha + \beta \\
     \psi_4 = \frac{1}{\sqrt{4}}(2p_{z2} - 2p_{z3} + 2p_{z5} - 2p_{z6}) \Longrightarrow E_4 = \alpha - \beta \\
      \psi_5 = \frac{1}{\sqrt{3}}(2p_{z1} - \frac{1}{2} 2p_{z2} - \frac{1}{2} 2p_{z3} + 2p_{z4} - \frac{1}{2} 2p_{z5} - \frac{1}{2} 2p_{z6}) \Longrightarrow E_5 = \alpha - \beta \\
      \psi_6 = \frac{1}{\sqrt{6}}(2p_{z1} - 2p_{z2} + 2p_{z3} - 2p_{z4} +  2p_{z5} -  2p_{z6}) \Longrightarrow E_6 = \alpha - 2\beta 
\end{split}
\end{equation}


\begin{table}[htb]
	\centering
	\caption{\label{qua:Quadro_1} Energias dos orbitais de fronteira de alguns hidrocarbonetos aromáticos.}	
	\begin{tabular}{lrrr}
		\toprule
		\textbf{Molécula} & \textbf{HOMO/(eV)} & \textbf{LUMO/(eV)} & \textbf{Gap/(eV)}
		\\ 
		\midrule
        benzeno & -12.808 & -8.282 & 4.452 \\
        naftaleno & -12.073 & -9.338 & 2.635 \\
        antraceno & -11.642 & -9.839 & 1.803 \\
        fenantreno & -12.023 & -9.310 & 2.713 \\
        azuleno & -11.730 & -9.872 & 1.858 \\
        pentaleno & -11.492 & -10.809 & 0.683 \\
        fulveno & -11.991 & -10.338 & 1.653 \\
        bifenileno & -11.555 & -9.553 & 2.002 \\
        tolueno & -12.502 & -8.348 & 4.154 \\
    \bottomrule
	\end{tabular}
	\fonte{Autor(a).}
\end{table}

\section{Hidrocarbonetos aromáticos}

Analisando os dados da \autoref{tab:energies}, concluímos que a ordem das estabilidades dos compostos está de acordo com aqueles apresentados por Hoffmann \autocite{Hoffmann1963}, e com os dados experimentais publicados por Heilbronner \autocite{ginsburg1959}. Por exemplo, o naftaleno mostra-se 32.3 kcal/mol mais estável do que o azuleno, o que é um ótimo resultado se comparado com o valor de referência (32.6 kcal/mol)\autocite{ginsburg1959}. O fulveno, por sua vez, é computado como sendo 25.6 kcal/mol menos estável do que o benzeno, em relação ao valor de 27 kcal/mol tomado como modelo\autocite{CHENG1956}. O antraceno, por sua vez, é 4.2 kcal/mol mais estável do que o fenantreno, enquanto o último isômero é, na verdade, 6.9 kcal/mol mais estável \autocite{Hoffmann1963}, o que provém do fato de que há repulsão estérica entre os hidrogênios do fenantreno.

\begin{table}[htb]
	\centering
	\caption{\label{tab:energies} Energias dos elétrons $\pi$ nos compostos aromáticos.}	
	\begin{tabular}{lrr}
		\toprule
		\textbf{Molécula} & $E_\pi / (eV)$ & $E_\pi / (kcal/mol)$
		\\ 
		\midrule
        Etileno & -26.336 & -609.628 \\
        Butadieno & -52.964 & -1221.38 \\
        Benzeno & -80.208 & -1849.64 \\
        Fulveno & -78.928 & -1820.12 \\
        Naftaleno & -133.676 & -3082.64 \\
        Azuleno & -132.934 & -3065.53 \\
        Fenantreno & -187.286 & -4318.92 \\
        Antraceno & -187.020 & -4312.78 \\
    \bottomrule
	\end{tabular}
	\fonte{Autor(a).}
\end{table}

\section{Influência de K}\label{wolfsberg}


\section{Índices geométricos}

\subsection{Hidrocarbonetos policíclicos aromáticos}

Além dos orbitais moleculares e das funções de onda, o \textit{Balmy.jl} também permite analisar o \gls{HOMA}, um índice geométrico de aromaticidade. O arquivo de saída gerado pelo \textit{Balmy.jl} está representado na \autoref{fig:HOMA}. A ideia original e fundamental é que cada par de átomos pode ser envolvido em ambas as ligações $\sigma$ e $\pi$, com $R_\sigma > R_\pi$. A distância de ligação $R_i$ representa uma compressão de $R_\sigma$ e uma extensão de $R_\pi$. Ao observar os valores na \autoref{fig:HOMA}, nota-se que os sextetos mais aromáticos estão na região perférica mais externa (\gls{HOMA} $= 0.9$), o que é um fenômeno importante, pois o \gls{HOMA} conhecidamente superestima os valores de aromaticidade para sextetos periféricos\autocite{giov2020}. 

\begin{figure}[htb]
\caption{\label{fig:HOMA} Valores de HOMA para a estrutura do Kekuleno gerados com o \textit{Balmy.jl}.}
	\begin{center}
		\includegraphics[width=0.65\textwidth]{images/geom.png}
	\end{center}
	\fonte{Autor(a).}
\end{figure}

\begin{figure}[htb]
\caption{\label{fig:HOMA2} Valores de HOMA para a estrutura do Kekuleno gerados com o \textit{Balmy.jl}.}
	\begin{center}
		\includegraphics[width=1.0\textwidth]{images/fig2(6).png}
	\end{center}
	\fonte{Autor(a).}
\end{figure}

\subsection{Heterociclos aromáticos}

O efeito da ressonância em estruturas heterocíclicas depende da geometria do sistema e do tipo de heteroátomo presente no anel. No caso dos compostos que correspondem a anéis de cinco membros, como pirrol, furano, e seus derivados, são possíveis conjugações entre os pares de elétrons livres (do(s) heteroátomo(s)) e o sistema $\pi$. Consequentemente, são gerados híbridos de ressonância não equivalentes entre si, uma com separação de carga, e outra sem, não permitindo que a deslocalização eletrônica seja completa. Por outro lado, anéis de seis membros, a exemplo da piridina e seus derivados, apresentam conjugação $\pi-\pi$ total, com estruturas de ressonância que não têm separação de carga.

Após os efeitos de ressonância, os índices HOMED (Tabela 11) calculados para as estruturas otimizadas
do pirrol e do furano são
inferiores aos dos heteroaromáticos de seis membros, como a piridina e o íon piranil, respectivamente.
O índice HOMED para o íon piranilíaco também é menor do que o índice para a piridina. Uma substituição do íon CH
pelo átomo N-aza em anéis com cinco membros parece reduzir o índice HOMED em maior grau
para O- que para N-derivados. Para anéis com seis membros, presença do átomo adicional N-aza em azinas
não destrói o sistema de deslocalização completa de elétrons $\pi$ no sistema. Para os anéis de seis membros
O-derivados, o índice HOMED depende fortemente da posição do átomo N-aza. Ele diminui
para posições 3 e 3,5; onde o átomo N pode estar próximo do átomo C+, e aumenta para a posição 4 onde o átomo N pode assumir a carga positiva.
% ---

% 5 - Conclusão
% ---
%\phantompart
\chapter{Conclusão}
% ----------------------------------------------------------

Por meio deste trabalho, foi apresentado o \textit{Balmy.jl}, uma aplicação \textit{web} capaz de auxiliar no cálculo de orbitais moleculares usando o método de Hueckel estendido. Comprovou-se, através dos resultados obtidos, que as energias dos orbitais moleculares foram coerentes com a literatura original de Roald Hoffmann, criador da metodologia \gls{EHMO}. Os tempos de execução dos cálculos em comparação a um código já existente (\gls{YAeHMOP}) foram extremamente satisfatórios, com ganhos de 12.5\% em termos de desempenho e performance conforme aumentamos o número de bases no sistema (no caso, foi testado o poletileno). Isso ocorre devido às otimizações que a linguagem Julia possui para implementação de equações matemáticas em relação a outras linguagens, como C, por exemplo. Além disso, outros ganhos foram observados no uso do \textit{Balmy.jl} se comparado ao \gls{YAeHMOP}, como a implementação com interface gráfica, o acesso via navegador \textit{web}, e os métodos de avaliação de compostos aromáticos (que não estão presentes no \gls{YAeHMOP}). Tudo isso torna a experiência do usuário ainda mais completa quando ele utiliza o \textit{Balmy.jl}.

Além dos cáculos via \gls{EHMO} feitos no \textit{software}, os índices de quantificação da aromaticidade sob o ponto de vista geométrico também mostraram resultados que estão em bom acordo com a literatura, comprovando que toda a implementação foi realizada adequadamente. A abordagem estatística do índice de aromaticidade mostra como o \gls{HOMA} pode ser adotado enquanto o melhor descritor de aromaticidade geométrica, pois ele é baseado em uma fundamentação físico-química das energias e forças de ligação obtidas pela teoria do oscilador harmônico. Apesar disso, o índice \gls{HOMA} não é um bom descritor de aromaticidade para sistemas heteroaromáticos. Nesse sentido, o \textit{Balmy.jl} também contém o \gls{rHOMA} e o \gls{HOMED} implementados para análise de sistemas heterocíclicos aromáticos. De acordo com os resultados mostrados, o \gls{HOMED} é um índice de heteroaromaticidade mais adequado do que o \gls{rHOMA}, devido às parametrizações do comprimento ótimo de ligação levando em conta sistemas aromáticos mínimos, como a metilamina e metilimina, por exemplo.

E foi nesse contexto do estudo da aromaticidade em diferentes famílias de compostos que surgiu o \textit{Balmy.jl}, uma ferramenta publicada na \textit{web} com o objetivo de auxiliar na interpretação eletrônica e geométrica desse fenômeno. O \textit{software} está disponível de forma gratuita, e por ser acessível tal qual uma página na \textit{internet}, não apresenta os mesmos problemas de compatibilidade do que uma aplicação \textit{desktop} com relação a diferentes sistemas operacionais. Além disso, o usuário não precisará se incomodar com atualizações em sua máquina local, pois tudo está hospedado em um servidor remoto, onde as requisições e cálculos são processados. Além disso, a implementação com a interface gráfica permite aos usuários menos experientes utilizar o \textit{Balmy.jl} de maneira extremamente fácil, sem nenhum requisito quanto a conhecimento de linguagens de programação ou códigos de linha de comando. Isso é novo com a relação a outros \textit{softwares} existentes dentro do mundo da química computacional, que não acumulam em si todas as vantagens que o \textit{Balmy.jl} demonstra considerando os métodos implementados.
% ---

% ----------------------------------------------------------
% ELEMENTOS PÓS-TEXTUAIS
% ----------------------------------------------------------
\postextual
% ----------------------------------------------------------

% ----------------------------------------------------------
% Referências bibliográficas
% ----------------------------------------------------------
\begingroup
    \printbibliography[title=REFERÊNCIAS]
\endgroup

% ----------------------------------------------------------
% Glossário
% ----------------------------------------------------------
%
% Consulte o manual da classe abntex2 para orientações sobre o glossário.
%
%\glossary

% ----------------------------------------------------------
% Apêndices
% ----------------------------------------------------------

% ---
% Inicia os apêndices
% ---
\begin{apendicesenv}
%	\partapendices* 
	% ----------------------------------------------------------
\chapter{Método de Hueckel}
% ----------------------------------------------------------

Considerando métodos semiempíricos fundamentados na teoria \gls{HF}, utilizam-se as devidas aproximações resultantes dos conceitos e dados experimentais, pois estes tendem a ser intuitivos. Em alcenos e alcinos, os elétrons-$\pi$ estão presentes nos orbitais p, os quais são considerados como independentes da estrutura sigma dos orbitais híbridos e elétrons sigma. Funções de onda $\psi$ é dada pela \autoref{ap:eq:1}.

\begin{figure}[htb]
    \vspace{2\baselineskip}
\begin{equation}
    \label{ap:eq:1}
    \psi =  a_1\tikzmarknode{wavefunction}{\highlight{blue}{$\phi_1$}} + \tikzmarknode{coefficient}{\highlight{red}{$a_2$}}\phi_2 + \dots + a_i \phi_i
\end{equation}
\begin{tikzpicture}[overlay,remember picture,>=stealth,nodes={align=left,inner ysep=1pt},<-]
    \path (wavefunction.north) ++ (-1.2em,1.5em) node[anchor=south east,color=blue!67] (scalep){\textbf{função de onda}};
    \draw [color=blue!87](wavefunction.north) |- ([xshift=-3em,color=blue]scalep.south west);
    
    \path (coefficient.south) ++ (1.2em,-1.5em) node[anchor=north west,color=red!67] (scalep){\textbf{coeficiente de contribuição}};
    \draw [color=red!87](coefficient.south) |- ([xshift=3em,color=red]scalep.north east);
\end{tikzpicture}
\end{figure}

Como somente os elétrons dos orbitais p estão contribuindo para a função de onda, a \autoref{ap:eq:1} pode ser escrita como a \autoref{ap:eq:2}.

\begin{figure}[htb]
    \vspace{2\baselineskip}
\begin{equation}
    \label{ap:eq:2}
    \psi = a_1 p_1 + a_2 p_2 + \dots + a_i p_i
\end{equation}
\end{figure}

Tomando o eteno como exemplo, pode-se afirmar que cada carbono contribui com um elétron para a ligação $\pi$, sendo $p_1$ e $p_2$ os elétrons dos átomos de carbono 1 e 2, cujas respectivas contribuições são ponderadas por $a_1$ e $a_2$. No caso dos elétrons p não hibridizados, os orbitais moleculares são formados pela \gls{LCAO} $p_1$ e $p_2$. Sobreposição entre orbitais atômicos podem ocorrer de forma simétrica (resultando em um orbital ligante) ou antissimétrica (formando um orbital antiligante).

\begin{figure}[htb]
    \vspace{2\baselineskip}
\begin{equation}
    \label{ap:eq:3}
    \begin{cases}
\tikzmarknode{simetric}{\highlight{blue}{$\psi^+$}} = a_1 p_1 + a_2 p_2 \\
\tikzmarknode{antissimetric}{\highlight{red}{$\psi^-$}} = a_1 p_1 - a_2 p_2
    \end{cases}
\end{equation}
\begin{tikzpicture}[overlay,remember picture,>=stealth,nodes={align=left,inner ysep=1pt},<-]
    \path (simetric.north) ++ (-1.2em,1.5em) node[anchor=south east,color=blue!67] (scalep){\textbf{função simétrica}};
    \draw [color=blue!87](simetric.north) |- ([xshift=-3em,color=blue]scalep.south west);
    
    \path (antissimetric.south) ++ (1.2em,-1.5em) node[anchor=north west,color=red!67] (scalep){\textbf{função antissimétrica}};
    \draw [color=red!87](antissimetric.south) |- ([xshift=3em,color=red]scalep.north east);
\end{tikzpicture}
\end{figure}

Usando a \autoref{ap:eq:4} (equação de Schroedinger) como base fundamental para avançar na discussão do cálculo dos valores de energia associados aos orbitais, pode-se aplicar o princípio variacional como uma aproximação necessária e conveniente para encontrar as funções de minimização da energia relativa aos orbitais no estado fundamental. Este método consiste em escolher uma função de onda inicial que dependa de um ou mais parâmetros, e encontrar os valores destes parâmetros para cada valor esperado onde a energia seja a menor possível. A função de onda obtida na substituição dos parâmetros pelos valores encontrados será uma aproximação do estado fundamental da função de onda, e o valor esperado de energia neste estado será majorante para a energia deste estado fundamental.

%%% TODO: explicar o princípio variacional 

\begin{figure}[htb]
    \vspace{2\baselineskip}
\begin{equation}
\label{ap:eq:4}
    \hat{H} \psi = \tikzmarknode{energy}{\highlight{blue}{E}} \psi
\end{equation}
\begin{tikzpicture}[overlay,remember picture,>=stealth,nodes={align=left,inner ysep=1pt},<-]
    \path (energy.north) ++ (-1.2em,1.5em) node[anchor=south east,color=blue!67] (scalep){\textbf{energia}};
    \draw [color=blue!87](energy.north) |- ([xshift=-3em,color=blue]scalep.south west);
\end{tikzpicture}
\end{figure}

\begin{equation}
\label{ap:eq:5}
    \psi \hat{H} \psi = \psi^2 E
\end{equation}


Multiplicando ambos os lados por $\psi$, obtém-se a \autoref{ap:eq:5}. Aplicado a integral nessa expressão (mantendo a igualdade) e, em seguida, isolando a energia $E$, obtém-se a \autoref{ap:eq:6}.

\begin{equation}
\label{ap:eq:6}
    E = \frac{\displaystyle \int \psi \hat{H} \psi d\tau}{\displaystyle \int \psi^2 d\tau}
\end{equation}

Na \autoref{ap:eq:5}, a energia esperada vai ser superestimada em relação ao valor real. Desse modo, o resultado calculado vai ser minimizado através de um procedimento matemático submetido a um conjunto de funções de base. No caso do eteno, podemos substituir o termo $\psi$ da \autoref{ap:eq:6} pela \autoref{ap:eq:2}, obtendo a \autoref{ap:eq:7}.

\begin{figure}[htb]
    \vspace{2\baselineskip}
\begin{equation}
\label{ap:eq:7}
    E = \frac{\displaystyle \int \tikzmarknode{LCAO}{\highlight{blue}{$(a_i p_i + a_2 p_2)$}} \hat{H} \highlight{blue}{$(a_i p_i + a_2 p_2)$} d\tau}{\displaystyle \int \highlight{blue}{$(a_i p_i + a_2 p_2)$}^2 d\tau}
\end{equation}
\begin{tikzpicture}[overlay,remember picture,>=stealth,nodes={align=left,inner ysep=1pt},<-]
    \path (LCAO.north) ++ (-1.2em,1.5em) node[anchor=south east,color=blue!67] (scalep){\textbf{LCAO}};
    \draw [color=blue!87](LCAO.north) |- ([xshift=-3em,color=blue]scalep.south west);
\end{tikzpicture}
\end{figure}

Abrindo a expressão da \autoref{ap:eq:7}, é possível isolar as seguintes integrais:

\begin{figure}[htb]
    \vspace{2\baselineskip}
\begin{equation}
\label{ap:eq:8}
    \displaystyle \int (p_1 \hat{H} p_1) d\tau = \tikzmarknode{alpha}{\highlight{blue}{$\alpha$}} \\
\end{equation}
\begin{tikzpicture}[overlay,remember picture,>=stealth,nodes={align=left,inner ysep=1pt},<-]
    \path (alpha.north) ++ (-1.2em,1.5em) node[anchor=south east,color=blue!67] (scalep){\textbf{integral de Coulomb}};
    \draw [color=blue!87](alpha.north) |- ([xshift=-3em,color=blue]scalep.south west);
\end{tikzpicture}
\end{figure}

A integral de Coulomb, mostrada na \autoref{ap:eq:8}, é o Hamiltoniano para a energia de Coulomb relativa a um elétron com uma função de onda $p_i$ no campo do átomo $i$ e influenciado pelo seu núcleo. Esse elétron não sofre efeito de nenhum outro núcleo. Essa aproximação funciona melhor quando os átomos das vizinhanças não possuem cargas elétricas. Ou seja, $\alpha$ é uma função da carga nuclear e do tipo de orbital. Por ser um termo atrativo, possui um valor negativo. 

\begin{figure}[htb]
    \vspace{2\baselineskip}
\begin{equation}
\label{ap:eq:9}
    \displaystyle \int (p_1 \hat{H} p_2) d\tau = \displaystyle \int (p_2 \hat{H} p_1) d\tau = \tikzmarknode{beta}{\highlight{blue}{$\beta$}}
\end{equation}
\begin{tikzpicture}[overlay,remember picture,>=stealth,nodes={align=left,inner ysep=1pt},<-]
    \path (beta.north) ++ (-1.2em,1.5em) node[anchor=south east,color=blue!67] (scalep){\textbf{integral de troca (ou de ressonância)}};
    \draw [color=blue!87](beta.north) |- ([xshift=-3em,color=blue]scalep.south west);
\end{tikzpicture}
\end{figure}

As integrais de sobreposição são dadas pelas expressões na \autoref{ap:eq:10} e na \autoref{ap:eq:11}. Se $i=j$, então $S_{ii} = \displaystyle \int p_i p_i d \tau$ para os orbitais atômicos normalizados. Se $i \neq j $, a integral de sobreposição é igual a $0$ para os orbitais atômicos ortogonais. Ou seja, o valor de $S$ varia de $0$ a $1$ e representa uma medida da não-ortogonalidade dos orbitais. Uma vez que as funções $p$ dos orbitais são amplamente separadas no espaço e são independentes, espera-se que elas sejam ortogonais.

\begin{figure}[htb]
    \vspace{2\baselineskip}
\begin{equation}
\label{ap:eq:10}
    \displaystyle \int (p_1 p_1) d\tau = \tikzmarknode{s11}{\highlight{blue}{$S_{11}$}} = \displaystyle \int (p_2 p_2) d\tau = \highlight{blue}{$S_{22}$}
\end{equation}
\begin{tikzpicture}[overlay,remember picture,>=stealth,nodes={align=left,inner ysep=1pt},<-]
    \path (s11.north) ++ (1.2em,1.5em) node[anchor=south west,color=blue!67] (scalep){\textbf{integral de sobreposição}};
    \draw [color=blue!87](s11.north) |- ([xshift=3em,color=blue]scalep.south east);
\end{tikzpicture}
\end{figure}

\begin{figure}[htb]
    \vspace{2\baselineskip}
\begin{equation}
\label{ap:eq:11}
    \displaystyle \int (p_1 p_2) d\tau = \tikzmarknode{s12}{\highlight{blue}{$S_{12}$}} = \displaystyle \int (p_2 p_1) d\tau = \highlight{blue}{$S_{21}$}
\end{equation}
\begin{tikzpicture}[overlay,remember picture,>=stealth,nodes={align=left,inner ysep=1pt},<-]
    \path (s12.north) ++ (1.2em,1.5em) node[anchor=south west,color=blue!67] (scalep){\textbf{integral de sobreposição}};
    \draw [color=blue!87](s12.north) |- ([xshift=3em,color=blue]scalep.south east);
\end{tikzpicture}
\end{figure}

Com as simplificações feitas anteriormente, é possível reescrever a expressão que calcula a energia para a molécula de eteno de acordo com a \autoref{ap:eq:12}.

\begin{figure}[htb]
    \vspace{2\baselineskip}
\begin{equation}
\label{ap:eq:12}
    E = \frac{a_1^2 \alpha + 2a_1 a_2 \beta + a_2^2 \alpha}{a_1^2 S_{11} + 2a_1 a_2 S_{12} + a_2^2 S_{22}}
\end{equation}
\end{figure}

Desse modo, conhecendo $\alpha$, $\beta$ e $S$, a energia pode ser calculada. O critério de minimização em relação a alguns parâmetros é mostrado na \autoref{ap:eq:12}.

\begin{figure}[htb]
    \vspace{2\baselineskip}
\begin{equation}
\label{ap:eq:12}
    \frac{\partial E}{\partial a_1} = \frac{\partial E}{\partial a_2} = 0
\end{equation}
\end{figure}

Como alternativa, ao invés de variar a função teste para encontrar o valor mínimo de $E$, pode-se variar os coeficientes lineares.
\end{apendicesenv}
% ---


% ----------------------------------------------------------
% Anexos
% ----------------------------------------------------------

% ---
% Inicia os anexos
% ---
\begin{anexosenv}
%	\partanexos*
	\input{aftertext/anexo_a}
\end{anexosenv}

%---------------------------------------------------------------------
% INDICE REMISSIVO
%---------------------------------------------------------------------
%\phantompart
%\printindex
%---------------------------------------------------------------------

\end{document}
